\addcontentsline{toc}{chapter}{Conclusions}
\markboth{Conclusions}{Conclusions}
In this thesis we studied the dissipation of primordial tensor perturbations, or primordial gravitational waves, into CMB spectral distortions. On their own, spectral distortions constitute a probe of primordial perturbations at smaller scales than those of CMB anisotropies or galaxies surveys - $k\sim10^{-3} - 1$ Mpc$^{-1}$ -, which can still be studied in the linear regime. Spectral distortions generated by the dissipation of gravitational waves are at even smaller scales ($k\sim10^{8}$ Mpc$^{-1}$) than those originated by scalars, i.e density perturbations. Although leading to a signal smaller than scalar perturbations, CMB spectral distortions generated by tensor modes are a pristine footprint from the early universe complementary to the B-mode polarization of the CMB anisotropy pattern.

For the purposes of this work, we developed a modified version of \texttt{CLASS} (Cosmic Linear Anisotropies Solving System) which can compute CMB spectral distortions originated from tensor dissipation, for different cosmological setup. Our implementation further improves publicly available codes (e.g. the code used in \cite{Kite_2021}) allowing for the Green's function method \cite{Chluba_Green}, which computes numerically the exact plasma heating history, to be used with primordial gravitational waves for the first time. For this reason, our implementation takes into account also the dissipation occurring during matter domination, which is usually neglected in most of the previous works.

After having reproduced the results obtained by J. Chluba et al. in \cite{Chluba_tens_diss}, we showed that allowing for spectral distortions to be sourced during a smooth $\mu$-y transition yields $\mu$ distortions of almost the double the amplitude of those sourced considering energy injection only in the radiation dominated era. Furthermore, we showed that the dissipation occurring during the matter dominated era contributes in a non-negligible way to the y-distortions window function. We also studied how a bump in the tensor power spectrum can affect spectral distortions:we found that a log-normal template with amplitude $\mathcal A_\text{bump}\sim10^{-2}-10^{-4}$ can push the $\mu$-distortions in a detectable regime only if they are wide enough, $\sigma_\text{bump}\sim 0.1-1$. We also observed that as the log-normal gets wider ($\sigma_\text{bump}\sim10$), spectral distortions plateau at quite high amplitudes ($\mu\sim10^{-6}-10^{-8}$ and $y\sim10^{-5}-10^{-7}$). However, this last scenario is definitely ruled-out by current constraints at the CMB anisotropies scales.

We concluded this work by demonstrating the synergy between CMB anisotropies and spectral distortions for primordial gravitational waves. For this we conducted a Monte Carlo Markov Chains analysis of how CMB spectral distortions measurements from future space missions could complement the state of the art of B-mode polarization data, such as BICEP/\emph{Keck} 2018. We studied a power spectrum described by a power law using a two scales parametrization, in which we trade the spectral index for a second value of the tensor-to-scalar ratio. On assuming flat priors for both tensor-to-scalar ratios and choosing as reference scales $k_1=0.02$ Mpc$^{-1}$ and $k_2=0.005$ Mpc$^{-1}$, we showed that a PIXIE-like experiment, in the case of a non detection of spectral distortions, will be able to improve the constraints set by BICEP/\emph{Keck} alone ($r_{0.005}<0.060$ and $r_{0.02}<0.103$ at 95\% CL) to $r_{0.005}<0.072$ and $r_{0.02}<0.072$ at 95\% CL or to $r_{0.005}<0.076$ and $r_{0.02}<0.65$ at 95\% CL, assuming a perfect knowledge of the foreground. We also presented these results as constraints on the spectral index and the tensor-to-scalar ratio at an intermediate scale $k=0.01$ Mpc$^{-1}$: in this way the role of future spectral distortions measurements in probing tensor spectra with blue tilt becomes more evident. We also conducted a MCMC analysis of a log-normal bump with fixed width ($\sigma_\text{bump}$=0.4) at the fixed scale $k_\text{pk}=10$ Mpc$^{-1}$. For this analysis we assumed a non-detection outcome of a PIXIE-like experiment, as fiducial model, which resulted in the following constraint on the bump amplitude: $\mathcal A_\text{bump}<3.2\times10^{-2}\ 95\%$CL. Considering instead the assumption of a fixed foreground or with free $y_\text{reio}$, the constraint becomes tighter, respectively $\mathcal A_\text{bump}<7.7\times10^{-5}\ 95\%$CL and $\mathcal A_\text{bump}<4.4\times10^{-4}\ 95\%$CL.

The innovative results of this work can be improved further following different direction.
From the point of view of theoretical predictions, a more precise heating rate must be found: although the final effect is expected to be almost negligible, using a smoother transition between radiation and matter domination, we should be able to remove all the discontinuities in the window function, obtaining more precise amplitudes. It would be interesting to understand if this improvement applies to the scalar sector as well. \\
The most relevant result we obtained are the improved constraints on the tensor-to-scalar ratio combining BICEP/\emph{Keck} 2018 B-mode polarization and PIXIE-like data: the natural next step is to include  \emph{Planck} T, E data by also allowing the $\Lambda$CDM parameters to vary. It would be also interesting to study other space experiments, such as the LiteBIRD \cite{LiteBIRD:2022cnt} satellite dedicated to B-mode polarization and the proposal FOSSIL\footnote{At the time this thesis was submitted, i.e. on Oct. 7, 2025, it was announced that FOSSIL passed to the step 2 phase of the ESA M8 call.} for spectral distortions submitted to the ESA M8 call.\\
We have explored the relevance of a Pixie-like experiment in setting upper limits to primordial gravitational waves alone and in combination with CMB anisotropies. It will be interesting to study the capability of detection for a non-zero $\mu$ from gravitational waves and its possible synergy with the pulsar timing array recent measurements.\\
Our modified version of \texttt{CLASS} also allows for exotic sources of primordial gravitational waves to be considered, for example primordial magnetic fields, as we explained in Section \ref{sec:primordial_magn}. This could open for the possibility of exploiting tensor-induced spectral distortions as a probe of these phenomena.\\
All these paths are currently under consideration for future publications during my PhD. 



