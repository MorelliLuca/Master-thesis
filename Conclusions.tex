\addcontentsline{toc}{chapter}{Conclusions}
\markboth{Conclusions}{Conclusions}
In this thesis we studied the dissipation of primordial gravitational waves as a sourcing mechanism for CMB spectral distortions. On their own, spectral distortions constitute a probe of primordial perturbations at smaller scales beyond those of CMB anisotropies or galaxies surveys. Spectral distortions generated by the dissipation of primordial gravitational waves push the probed range of wave numbers to even smaller scales ($k\sim10^{8}$ Mpc$^{-1}$). Moreover, the expected distortions sourced by this mechanism, due to their strong synergy with the CMB anisotropy data, can be used to further constraint the tensor primordial power spectrum improving previous upper limits on the tensor-to-scalar ratio.

For the purposes of this work, we developed a modified version of \texttt{CLASS} (Cosmic Linear Anisotropies Solving System) which can compute the amplitudes of primordial gravitational waves-induced spectral distortions for different cosmological setup. Our implementation further improves publicly available codes (e.g. the code used in \cite{Kite_2021}) allowing for the Green's function method \cite{Chluba_Green}, which computes numerically the exact plasma heating history, to be used with primordial gravitational waves for the first time. For this reason, our implementation takes into account also the dissipation occurring during matter domination, which is usually neglected in most of the previous works.

After having reproduced the results obtained by J. Chluba et al. in \cite{Chluba_tens_diss}, we showed that allowing for spectral distortions to be sourced during a smooth $\mu$-y transition yields $\mu$ distortions of almost the double the amplitude of those sourced considering energy injection only in the radiation dominated era. Furthermore, we showed that the dissipation occurring during the matter dominated era contributes in a non-negligible way to the y-distortions window function. We also studied how a bump in the tensor power spectrum can affect spectral distortions: we found that a log-normal template with amplitude $\mathcal A_\text{bump}\sim10^{-2}-10^{-4}$ can push the $\mu$-distortions in a detectable regime only if they are wide enough, $\sigma_\text{bump}\sim 0.1-1$. We also observed that as the log-normal gets wider ($\sigma_\text{bump}\sim10$) spectral distortions plateau at quite high amplitudes ($\mu\sim10^{-6}-10^{-8}$ and $y\sim10^{-5}-10^{-7}$). However, this last scenario is definitely ruled-out by current constraints at the CMB scales.

We concluded this work showing the power of joint analysis of CMB anisotropies data and spectral distortions. For this we conducted a Monte Carlo Markov Chains analysis using data from the BICEP/\emph{Keck} 2018 data release and a PIXIE-like simulated experiment. We studied a power law power spectrum using a two scales parametrization, in which we trade the spectral index for a second value of the tensor-to-scalar ratio. Assuming flat priors for both and choosing as reference scales $k_1=0.02$ Mpc$^{-1}$ and $k_2=0.005$ Mpc$^{-1}$, we showed that a PIXIE-like experiment, in the case of a non detection of spectral distortions, is able to improve the constraints set by BICEP/\emph{Keck} alone ($r_{0.005}<0.060$ and $r_{0.02}<0.103$ at 95\% CL) to $r_{0.005}<0.072$ and $r_{0.02}<0.072$ at 95\% CL or to $r_{0.005}<0.076$ and $r_{0.02}<0.65$ at 95\% CL, assuming a perfect knowledge of the foreground. We also translated these results into constraints on the spectral index and the tensor-to-scalar ratio at an intermediate scale $k=0.01$ Mpc$^{-1}$. This showed that spectral distortions improve constraints on the tilt of the power spectrum, moving the spectral index from a slightly blue tilt ($n_t=0.57^{+2.39}_{-2.66}$ at 95\%CL with only BICEP/\emph{Keck} 2018) to a slightly more red one $n_t=-0.13^{+1.48}_{-2.31}$ at 95\% CL or $n_t=-0.27^{+1.52}_{-2.17}$ at 95\% CL with fixed foreground. On the other hand, the constraints on the tensor-to-scalar ratio did not improve, moving from $r_{0.01}=0.022^{+0.021}_{-0.0019}$ 95\%CL with BICEP\/emph{Keck} 2018 alone, to $r_{0.01}=0.024^{+0.022}_{-0.0021}$ 95\%CL with BICEP\/emph{Keck} 2018 + PIXIE. This shows that spectral distortions improve these constraints by ruling-out blue power spectra. \\ We also conducted a MCMC analysis of a lognormal bump with fixed width ($\sigma_\text{bump}$=0.4) at the fixed scale $k_\text{pk}=10$ Mpc$^{-1}$. For this analysis we assumed a non-detection outcome of a PIXIE-like experiment, as fiducial model. This analysis resulted in the following constraint on the bump amplitude: $\mathcal A_\text{bump}<3.2\times10^{-2}\ 95\%$CL. Considering instead the assumption of a fixed foreground or with free $y_\text{reio}$, the constraint becomes tighter, respectively $\mathcal A_\text{bump}<7.7\times10^{-5}\ 95\%$CL and $\mathcal A_\text{bump}<4.4\times10^{-4}\ 95\%$CL.

The innovative results of this work can be improved further following different direction. The most relevant result we obtained are the improved constraints on the tensor-to-scalar ratio combining BICEP/\emph{Keck} 2018 and PIXIE-like data: the natural next step is to exploit this synergy and include also data from the \emph{Planck} satellite and constraint all the $\Lambda$CDM parameters.\\
Moreover, better constraints on a bump at the spectral distortions scales can be obtained by freeing also the width and the position of the bump and exploring which subset of the parameter space of the amplitude and these two parameters leads to a possible detection of tensor-induced spectral distortions.\\
Our modified version of \texttt{CLASS} also allows for exotic sources of primordial gravitational waves to be considered, for example primordial magnetic fields, as we explained in Section \ref{sec:primordial_magn}. This could open for the possibility of exploiting tensor-induced spectral distortions as a probe of these phenomena.\\
Lastly, a more precise heating rate must be found: although the final effect is expected to be almost negligible, using a smoother transition between radiation and matter domination, we should be able to remove all the discontinuities in the window function, obtaining more precise amplitudes.\\
All these paths are currently under consideration for future publications during my PhD. 



