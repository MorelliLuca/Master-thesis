\addcontentsline{toc}{chapter}{Introduction}
\markboth{Introduction}{Introduction}
Thirty years after the discovery of the Cosmic Microwave Background radiation (CMB) in 1965 by Penzias and Wilson \cite{1965ApJ...142..419P}, the COBE/FIRAS mission \cite{COBE1996} successfully mapped the spectrum of CMB photons coming from all directions of the sky. While this represented the first of a series of increasingly precise measurements of the temperature anisotropies of the CMB, the instrument FIRAS showed that the CMB is a perfect blackbody gas of photons at the temperature of $T=(2.7255\pm0.0006)$ K \cite{COBE1996}. This result, which is in agreement with the prediction of the Hot Big Bang model for which at high redshifts the primordial plasma was entirely thermalized, was never further improved, leaving unanswered the possibility of small deviations from the blackbody spectrum, known as spectral distortions.\\ In the last decades, some missions, such as PIXIE \cite{pixie} and FOSSIL \cite{IAS_Fossil}, proposed to measure the CMB spectrum with a precision never reach before, opening for the possibility of a spectral distortion detection.

As the universe cooled down, the interaction rates of particles composing the primordial plasma decreased, making thermalization processes less efficient. At the highest redshifts $z>10^{6}$, Compton and double Compton scatterings and Bremsstrahlung were very efficient and, transferring energy between different species or producing new photons, they were able to always restore a perfect blackbody distribution of photons. At lower redshift, $10^{4}<z<10^{6}$, double Compton scattering and Bremsstrahlung became inefficient causing the number of photons to remain constant. In this era, called $\mu$-era, any energy injection in the plasma would perturb equilibrium and produce, though Compton scatterings, a Bose-Einstein distribution of photons with a non zero-chemical potential, which is interpreted as a $\mu$-distortion. At even lower redshifts $z<10^{-4}$, Compton scatterings became insufficient to drive back the plasma to an equilibrium state. In this last era, called y-era, any energy injection would directly distort the otherwise blackbody spectrum sourcing y-distortions.

In general, many processes can inject energy in the plasma sourcing spectral distortions, for example decaying Dark Matter or the Hawking radiation resulting from primordial black holes \cite{Lucca_2020}. This thesis we will focus on the phenomenon of dissipation of primordial perturbations: after quantum fluctuations are stretched to cosmological scales by inflation, as they re-enter the comoving Hubble horizon, they couple with the photon-baryon plasma sourcing anisotropies, such as the temperature anisotropies of the CMB. As the mean free path of photons increases, photons coming from patches of plasma at slightly different temperatures mix together producing a y-distortion. If this happens during the $\mu$-era, Compton scattering redistribute the energy among the photons, turning the y-distortion into a $\mu$-distortion.\\

Studying the resulting distortions is particularly interesting because they provide a unique probe of primordial perturbations on a range of scales ($k\sim50-10^{4}$ Mpc$^{-1}$) much smaller than those accessible though CMB anisotropies of galaxy surveys \cite{chlubafuturestepscosmologyusing}. Both scalar and tensor perturbations can dissipate and source distortions, however, while the former have been widely studied in literature \cite{Lucca_2020,Chluba_2x2} and are expected to produce distortions that should meet the sensitivity of future experiments, the latter have been only partially studied \cite{Chluba_tens_diss} and constitute a probe of primordial perturbations at even smaller scales ($k>10^{4}$ Mpc$^{-1}$). Moreover, current observations never revealed the presence of primordial tensor perturbations \cite{Ade_2021,planck2018results}, only constraining the tensor-to-scalar ratio $r$ at the scale of the CMB anisotropies. For this reason, studying spectral distortions sourced by dissipation of tensor perturbations could reveal for the first time new details on the tensor sector of the primordial perturbations. However, these distortions are expected to be much smaller than those generated by scalar perturbations, making their detection rather challenging. 

This work is organized as follows: from Chapter \ref{chap:FRW} to \ref{chap:GW} the basics of the Hot Big Bang model are introduced alongside the theory of cosmological perturbations, with particular interest to primordial gravitational waves. In Chapter \ref{chap:anisotropies} the theory of CMB anisotropies is presented, while in Chapter \ref{chap:SpectralDistortions} we introduce the generalities of spectral distortions. In Chapter \ref{chap:dissipation} we focus on the mechanism of mixing of blackbodies and dissipation of primordial perturbations, studying in depth the case of primordial gravitational waves. At the end of this chapter we also present the numerical result that we produced modifying the Einstein-Boltzmann code \texttt{CLASS} \cite{CLASS} to include primordial tensor perturbations dissipation: our results are then compared with those obtained by J. Chluba et al. in \cite{Chluba_tens_diss} for $\mu$-distortions, showing a perfect agreement of the two implementations. In this chapter we also present new refined results obtained allowing for dissipation to occur during the $\mu$-y transition: we show that larger tensor modes ($k\sim 10^{-3}-10^{-1}$ Mpc$^{-1}$), that cross back the Hubble horizon during the transition era, can still contribute significantly, increasing the $\mu$-distortions by a factor 2 for nearly scale invariant power spectra. Similar results, that were not analyzed in literature, for y-distortions are also presented.\\
To conclude, in Chapter \ref{chap:constr} we present the results of a Markov Chain Monte Carlo (MCMC) analysis we performed to constraint the tensor power spectrum using both data from BICEP/Keck \cite{Ade_2021} and PLANCK \cite{planck2018results} and simulated data from a PIXIE-like experiment. In this analysis we initially considered a power law tensor power spectrum, that we parametrized through the tensor-to-scalar ratio $r$ at two different scales. Then, we allowed for bumps in the power spectrum of primordial gravitational waves to they can enhance the tensor-resulting distortions while preserving the constraints at the scale of CMB anisotropies.    