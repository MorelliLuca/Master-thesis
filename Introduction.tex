\addcontentsline{toc}{chapter}{Introduction}
\markboth{Introduction}{Introduction}
Thirty years after the discovery of the Cosmic Microwave Background radiation (CMB) in 1965 by Penzias and Wilson \cite{1965ApJ...142..419P}, the COBE/FIRAS mission \cite{COBE1996} successfully mapped the spectrum of CMB photons coming from all directions of the sky, leading to the award of the Nobel Prize to John C. Mather and George F. Smooth. While this represented the first of a series of increasingly precise measurements for the temperature anisotropies of the CMB, the instrument FIRAS also showed that the CMB is a perfect blackbody gas of photons at the temperature of $T=(2.7255\pm0.0006)$ K \cite{COBE1996}. This result, which is in agreement with the prediction of the Hot Big Bang model for which at high redshifts the primordial plasma was entirely thermalized, was little improved, leaving unanswered the possibility of small deviations from the blackbody spectrum, known as \textbf{spectral distortions}. In the last decades, some missions, such as PIXIE \cite{pixie} and FOSSIL \cite{IAS_Fossil}, has been proposed to measure the CMB spectrum with a precision never reach before \cite{Chluba_2021}, opening for the possibility of a spectral distortion detection.

As the universe cooled down, the interaction rates of particles composing the primordial plasma decreased, making thermalization processes less efficient. At the highest redshifts $z>10^{6}$, Compton and double Compton scatterings and Bremsstrahlung were very efficient and, transferring energy between different species or producing new photons, they were able to always restore the perfect blackbody distribution of photons. At lower redshift, $10^{4}<z<10^{6}$, double Compton scattering and Bremsstrahlung became inefficient causing the number of photons to remain constant. In this era, called $\mu$-era, any energy injection in the plasma would perturb equilibrium and produce, though Compton scatterings, a Bose-Einstein distribution of photons with a non zero-chemical potential, which is interpreted as a $\mu$-distortion. At even lower redshifts $z<10^{-4}$, Compton scatterings became insufficient to drive back the plasma to an equilibrium state. In this last era, called y-era, any energy injection would directly distort the photon blackbody spectrum sourcing y-distortions. These distortions, which we can recognize due to their different shapes in frequency, are then imprinted in the decoupled photons of the CMB.

Already in the standard $\Lambda$CDM model many processes can inject energy in the plasma sourcing spectral distortions, such as the
adiabatic cooling of electrons and baryons, photon emission and absorption during recombination and the Sunayev-Zeldovich effect occurring after reionization. Also many exotic phenomena could results in energy injection, i.e. decaying and interacting Dark Matter, the Hawking radiation resulting from primordial black holes \cite{Lucca_2020}. In this thesis we will focus on the phenomenon of dissipation of primordial perturbations: after quantum fluctuations are stretched to cosmological scales by inflation, as they re-enter the Hubble radius, interacting with the photon-baryon plasma they generate temperature and polarization anisotropies. As the mean free path of photons increases, photons coming from patches of plasma at slightly different temperatures mix together producing a y-distortion. If this happens during the $\mu$-era, Compton scattering redistribute the energy among the photons, turning the y-distortion into a $\mu$-distortion.

Studying the resulting distortions is particularly interesting because they provide a unique probe of primordial perturbations on a range of scales ($k\sim50-10^{4}$ Mpc$^{-1}$) much smaller than those accessible though CMB anisotropies of galaxy surveys \cite{chlubafuturestepscosmologyusing}. Both scalar (i.e. density) and tensor (i.e. gravitational waves) perturbations can dissipate and source distortions, however, while the former have been widely studied in literature \cite{Lucca_2020,Chluba_2x2}, there has been less interest for the latter \cite{Chluba_tens_diss}, although they probe primordial perturbations at even smaller scales ($k>10^{4}$ Mpc$^{-1}$). Scalar perturbations have been accurately characterized by CMB anisotropies measurement, whereas we have only upper limits on primordial tensor perturbations \cite{Ade_2021,planck2018results}, often expressed as constraints on the tensor-to-scalar ratio $r$ at scales of the CMB anisotropies. For this reason, spectral distortions sourced by dissipation of tensor perturbations have a significant discovery potential for early universe physics, although these distortions are expected to be smaller than those generated by scalar perturbations. In this work we will study these spectral distortions with different original results, from the development of a code devoted to the theoretical predictions of all the types of contributions to spectral distortions by tensor perturbations, to the scientific capabilities for future space experiments to constrain them.

This work is organized as follows: from Chapter \ref{chap:FRW} to \ref{chap:GW} the basics of the Hot Big Bang model are introduced alongside the theory of cosmological perturbations, with particular interest to primordial gravitational waves. In Chapter \ref{chap:anisotropies} the theory of CMB anisotropies is presented, while in Chapter \ref{chap:SpectralDistortions} we introduce the generalities of spectral distortions. In Chapter \ref{chap:dissipation} we focus on the mechanism of mixing of blackbodies and dissipation of primordial perturbations, studying in depth the case of primordial gravitational waves. At the end of this chapter we also present the numerical result that we produced modifying the Einstein-Boltzmann code \texttt{CLASS} \cite{CLASS} to include all the constribution of primordial tensor perturbations dissipation to spectral distortions: our results are then compared with those obtained by J. Chluba et al. in \cite{Chluba_tens_diss} for $\mu$-distortions, showing a perfect agreement of the two implementations. In this chapter we also present new refined results obtained allowing for dissipation to occur during the $\mu$-y transition: we show that larger tensor modes ($k\sim 10^{-3}-10^{-1}$ Mpc$^{-1}$), that cross back the Hubble horizon during the transition era, can still contribute significantly, increasing the $\mu$-distortions by a factor 2 for nearly scale invariant power spectra. Similar results, that were not analyzed in literature, for y-distortions are also presented.\\
To conclude, in Chapter \ref{chap:constr} we present the results of a Markov Chain Monte Carlo (MCMC) analysis to jointly constraint the tensor power spectrum using both CMB anisotropy data, i.e. from BICEP/\textit{Keck} \cite{Ade_2021}, and CMB spectral distortions' data, i.e. as expected from a PIXIE-like experiment. The forecasts for the PIXIE-like experiments include contaminant and systematic effects as foreground and nuisance parameters. In this analysis we consider a power spectrum for tensor perturbations, that we analyze through to two tensor-to-scalar ratios $r_1$ and $r_2$ at two different scales. Then, we forecast the capability of a PIXIE-like experiment to constrain a lognormal bump in the tensor power spectrum in the CMB spectral distortions scales.   