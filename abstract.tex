\vspace*{10pt}
\begin{center}
	\large\textbf{Abstract}\normalsize
\end{center}
\vspace*{10pt}
\addcontentsline{toc}{section}{Abstract}
\markboth{}{Abstract}
\begin{adjustwidth}{1cm}{1cm}
Primordial gravitational waves (PGWs) generated during inflation, can leave observable imprints on
the Cosmic Microwave Background (CMB) in the form of spectral distortions (SDs). These distortions,
sourced by inefficient thermalization processes in the primordial plasma, provide a unique probe of
primordial perturbations on smaller scales (wavenumbers $k \sim 50 - 10^{4}$ Mpc$^{-1}$) than those accessible
through CMB anisotropies or galaxy surveys. Moreover, distortions sourced by PGWs could allow us to
explore even smaller scales ($k > 10^{4}$ Mpc$^{-1}$), uncovering never tested before early universe phenomena. However, the SDs resulting from PGWs are expected to be much smaller than those sourced by scalar perturbations or other mechanisms, making their detection rather challenging.

To investigate PGW-induced spectral distortions, we modified the \emph{distortion} module of \texttt{\texttt{CLASS}} to compute the heating rate associated to them, namely the rate at which energy is injected in the primordial plasma which then sources distortions. Our implementation enabled studying physical
scenarios not previously explored numerically, and it also allowed for the Green’s function method \cite{Chluba_Green}, which computes exactly the injection history during the $\mu$-y transition, to be used for the first time with PGWs. This showed that large scale PGWs ($k\sim 10^{-3}-10^{-1}$), that cross back the Hubble radius during the transition era, can still contribute to $\mu$-distortions, increasing them by a factor 2 for nearly scale invariant power spectra.

We also studied how a \emph{log-normal} bump in the PGWs power spectra, placed at the scales of the window function of $\mu$-distortions ($k_\text{bump}=10$ Mpc$^{-1}$) can influence SDs. We found that amplitudes of the order $\mathcal{A}_\text{bump}\sim 10^{-3}-10^{-2}$) are required to produce distortions comparable with scalar-induce ones, when the width of the bump ranges between $\sigma_\text{bump}\sim 0.1-1$. We also observed that increasing the width of the bump, as it starts to cover the whole $\mu$ window function, the SD amplitude saturates to a constant value, which depends on the amplitude of the bump. 

Finally, using simulated data from a PIXIE-like experiment, through a MCMC analysis we forecast constraints on the tensor power spectrum. Initially, we combined data from BICEP/\textit{Keck} 2018 and a PIXIE-like experiment to constraint the tensor-to-scalar ratio $r$ at two different scales: this showed that PGWs-induced spectral distortions can improve this kind of constraints from $r_{0.005}<0.060$ and $r_{0.02}< 0.103$ to $r_{0.005}<0.072$ and $r_{0.02}< 0.72$, both at 95\% CL. Then, using only PIXIE-like data, we constrained the amplitude of the lognormal bump at $\mathcal{A}_\text{bump}<3.2\times10^{-2} 95\%$CL, marginalizing over the foreground.

In this work we demonstrated for the first time that PGWs-induced spectral distortions can be used jointly with CMB polarization anisotropies to constraint the tensor primordial power spectrum. This synergy opens many new possibility for combining these data for example with \emph{Planck} data to futher improve the constraint on the tensor-to-scalar ratio.
\end{adjustwidth}