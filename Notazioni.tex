\section*{Notations and conventions}
\markboth{}{Notations and conventions}
\addcontentsline{toc}{section}{Notations and conventions}
\vspace*{\fill}
\textbf{Units:} This thesis mainly uses \emph{natural units} $c=\hbar=1$. Only when explicitly stated the the S.I. units are restored.\\  
\textbf{Indexes:} Greek indices $\mu,\nu,\rho,\dots$ denote spacetime components (0--3) of 4-vectors. Latin indices $i,j,k,\dots$ denote purely spatial components (1--3). The \emph{Einstein convention}, for which repeated indexes are summed over, is used. \\
\textbf{Metric:} The signature convention we chose is mostly plus $(-,+,+,+)$. In the Friedmann Robertson Walker metric $a$ is the \emph{scale factor}, $t$ the \emph{cosmic time} and $\tau$ the \emph{conformal time}. We always assume that we live in a \emph{flat universe} ($k=0$).\\
\textbf{Vectors:} the position 4-vector $x^\mu=(x^0,\mathbf x)$ is sometimes represented just by $x$, which should not be confused with the dimensionless frequency. 3-vectors are bold ($\mathbf v$) when the latin index is not written, 3-versos are bold with the upper hat symbol $\versor n$. The dot product $\mathbf{a\cdot b}$ is the euclidean product. \\
\textbf{Total derivatives:} Total derivatives w.r.t cosmic time $t$ are denoted by $\tfrac d{dt}$ or by a dot $\dot f(t)$, while $\tfrac d{d\tau}$ or a prime stands for total derivative w.r.t conformal time. Sometimes the prime is used for derivatives of function of a single variable, in this case the dependence is always indicated $f'(x)$.\\
\textbf{Partial and covariant derivatives:} Partial derivatives are denoted by $\partial_\mu\phi$ or by a comma $\phi_{,\mu}$, while covariant derivatives by the nabla symbol $\nabla_\mu\phi$ or by a semicolon $\phi_{;\mu}$. Bold nabla $\boldsymbol\nabla$ is instead used to denote the 3-dimensional rotor, divergence or Laplacian.
\textbf{Fourier transforms:} Fourier transformed functions are always denoted by a tilde $\tilde f(\mathbf k)$. The measure in momentum space, in our convention, is always divided by a factor $(2\pi)^3$.  \\
\textbf{Power spectra:} we use the convention for which $P_\delta(k)$ is the power spectrum of the generic perturbation $\delta$, while $\mathcal P_\delta(k)=k^3/(2\pi^2) P_\delta(k)$ is the dimensionless power spectrum. 