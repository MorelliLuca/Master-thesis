\label{app:MaxSymm}
\section{Killing vectors and symmetries of a mainifold}
Consider $\mathbb{R}^n$, this space is highly symmetric: it is isotropic and homogeneous, or, in a simpler way, every point and every direction "look" the same.\\
This means that $\mathbb{R}^n$ is symmetric under every rotation and translation: in $n$-dimensions there are $n$ possible translations (along the $n$ axes) and $n\frac{n-1}{2}$ possible rotations (for each axis we can rotate it towards $n-1$ other axes, but to avoid double counting we must divide by $2$), for a total number of symmetries equals to $$n+n\frac{n-1}{2}=n\frac{n+1}{2}.$$ 
An $n$-dimensional manifold is said to be \textbf{maximally symmetric} if it possesses the same number of symmetries of $\mathbb{R}^n$. In the differential geometry language, a symmetry is defined through isometries, that are diffeomorphisms under which the metric tensor is invariant.\\
For each symmetry of the metric we can define a \textbf{Killing vector}, which satisfies the Killing equation 
\begin{equation}
    0=(\pounds_{\vec K}g)_{\mu\nu}=\nabla_\mu K_\nu+\nabla_\nu K_\mu, \label{KillingEq}
\end{equation}
where $\pounds_{\vec{K}}$ is the Lie derivative along $\vec K$, which is the Killing vector.

We now want to show that a maximally symmetric space really possesses the maximum number of symmetries, namely the maximum number of independent\footnote{Linearly independent here means that $\not\exists$ a set of constants $c_n$ such that $$\sum_n c_n K_\mu^{(n)}(P)=0 \qquad \forall P\in\mathcal{M}.$$} Killing vectors.
Consider the action of the Riemann tensor applied on a 1-form
\begin{equation}
    \tensor{R}{^{\mu}_{\nu\rho\sigma}}K_\mu=-[\nabla_\rho,\nabla_\sigma]K_\nu,\label{RiemannDef}
\end{equation}
combing this with the algebraic Bianchi identity, ($\tensor{R}{^\mu_\nu_\rho_\sigma}+\tensor{R}{^\mu_\rho_\sigma_\nu}
+\tensor{R}{^\mu_\sigma_\nu_\rho}=0 $) implies that each Killing vector must satisfy
$$\nabla_\rho\nabla_\sigma K_\nu-\nabla_\sigma\nabla_\rho K_\nu +\nabla_\sigma\nabla_\nu K_\rho-\nabla_\nu\nabla_\sigma K_\rho+\nabla_\nu\nabla_\rho K_\sigma-\nabla_\rho\nabla_\nu K_\sigma=0.$$
This equation can be simplified by the Killing equation \eqref{KillingEq}, to then sum pairs of terms obtaining $$ 2(\nabla_\rho\nabla_\sigma K_\nu-\nabla_\sigma\nabla_\rho K_\nu -\nabla_\nu\nabla_\sigma K_\rho)=0,$$ that using \eqref{RiemannDef} turns out to be the following relation
\begin{equation}
    \tensor{R}{^\mu_{\nu\rho\sigma}}K_\mu=\nabla_\nu\nabla_\sigma K_\rho. \label{RNabla}
\end{equation}
This equation shows that the second covariant derivative acts on Killing vectors just as a linear application. In this way we can determine every derivative of a Killing vector in a specific point, just by knowing its value and the value of its first covariant derivative, at the same point.\\
If we now Taylor expand the Killing vector field around a point $P$, we will obtain an expansion that depends on the value in $P$ of all covariant derivatives of all orders, however we showed that we can evaluate those just knowing $K_\mu(P)$ and $\nabla_\nu K_\mu(P)$. This means that we can express the Killing vector field as a combination of two functions that do not depend on the Killing vector itself or on its derivatives:
\begin{equation*}
    K_\mu(x)=A_\mu\phantom{}^\lambda(x,P)K_\lambda(P)+B_\mu\phantom{}^{\lambda\nu}(x,P)\nabla_\nu K_\lambda(P),
\end{equation*}
these functions depend only on $x$, the point $P$, and the metric, through the Riemann tensor. For this reason these must be the same functions for all Killing vectors:
\begin{equation}\label{KillingExp}
    K_\mu^{(n)}(x)=A_\mu\phantom{}^\lambda(x,P)K_\lambda^{(n)}(P)+B_\mu\phantom{}^{\lambda\nu}(x,P)\nabla_\nu K_\lambda^{(n)}(P).
\end{equation}
The above equation tells us that a given Killing vector is determined by $K_\lambda^{(n)}(P)$, which has $N$ possible independent values, and by $\nabla_\nu K_\lambda^{(n)}(P)$, which has $N\frac{N-1}{2}$ independent values, due to its antisymmetry (which is a consequence of the Killing equation \eqref{KillingEq}).\\
In this way we have shown that the maximum number of independent Killing vectors in an $N$-dimensional manifold is exactly the same number that possesses $\mathbb{R}^N$ $$N+N\frac{N-1}{2}=N\frac{N+1}{2}.$$
\section{Riemann tensor in a maximally symmetric space}
We want to derive the form that has the Riemann tensor in a maximally symmetric space. Indeed, the existence of the maximum number of allowed independent Killing vector fields, allows for a  drastic simplification of the geometry of the manifold.\\
In general, equation \eqref{RNabla} must hold for every Killing vector, furthermore it also must be consistent with the commutator of covariant derivatives \eqref{RiemannDef}. This requirement and the fact that we have the maximum number of linearly independent Killing vectors will determine the form of $\tensor R{^\mu_{\nu\rho\sigma}}$. Consider \eqref{RiemannDef} applied to the two indices tensor $$ [\nabla_\sigma,\nabla_\nu]\nabla_\mu K_\rho=-R^\lambda_{\mu\sigma\nu}\nabla_\lambda K_\rho-R^\lambda _{\rho\sigma\nu}\nabla_\mu K_\lambda,$$ the equation \eqref{RNabla} can be used to obtain 
\begin{align*}
    \nabla&_\sigma(\tensor R{^\lambda_{\nu\rho\mu}}K_\lambda)-\nabla_\nu(\tensor R{^\lambda_{\sigma\rho\mu}}K_\lambda)=\\&=\nabla_\sigma \tensor R{^\lambda_{\nu\rho\mu}}K_\lambda-\nabla_\nu \tensor R{^\lambda_{\sigma\rho\mu}}K_\lambda+\ \tensor R{^\lambda_{\nu\rho\mu}}\nabla_\sigma K_\lambda- \tensor R{^\lambda_{\sigma\rho\mu}}\nabla_\nu K_\lambda=-\tensor R{^\lambda_{\mu\sigma\nu}}\nabla_\lambda K_\rho-\tensor R{^\lambda _{\rho\sigma\nu}}\nabla_\mu K_\lambda.
\end{align*}
Now, using the product rule on the left-hand side and the Killing equation \eqref{KillingEq} to move the index $\lambda$ to the covariant derivative in each term containing the covariant derivative of $K_\mu$, we get $$ (\nabla_\sigma \tensor R{^\lambda_{\nu\rho\mu}}-\nabla_\nu \tensor R{^\lambda_{\sigma\rho\mu}})K_\lambda=(\tensor R{^\lambda_{\nu\rho\mu}}\delta_{\sigma}\phantom{}^\alpha-R^\lambda_{\sigma\rho\mu}\delta_{\nu}\phantom{}^\alpha+\tensor R{^\lambda_{\mu\sigma\nu}}\delta_{\rho}\phantom{}^\alpha-\tensor R{^\lambda_{\rho\sigma\nu}}\delta_{\mu}\phantom{}^\alpha)\nabla_\lambda K_\alpha,$$
which holds for every Killing vector. Now, having the maximum number of independent Killing vectors, we can generate any other Killing vector from a combination of these. The general expansion \eqref{KillingExp} shows that a Killing vector field that vanishes in $P$, while its derivatives does not, can exists, and we surely can obtain it from a linear combination of the others. The above equation holds also for this one in $P$ only if the right-hand side vanishes too, this can happen only if the term in parentheses is symmetric in $\lambda\ \alpha$ (so that it vanishes when contracted with $\nabla_\lambda K_\alpha$ that is antisymmetric)$$\tensor R{^\lambda_{\nu\rho\mu}}\delta_{\sigma}\phantom{}^\alpha-\tensor R{^\lambda_{\sigma\rho\mu}}\delta_{\nu}\phantom{}^\alpha+\tensor R{^\lambda_{\mu\sigma\nu}}\delta_{\rho}\phantom{}^\alpha-\tensor R{^\lambda_{\rho\sigma\nu}}\delta_{\mu}\phantom{}^\alpha=\tensor R{^\alpha_{\nu\rho\mu}}\delta_{\sigma}\phantom{}^\lambda-\tensor R{^\alpha_{\sigma\rho\mu}}\delta_{\nu}\phantom{}^\lambda+\tensor R{^\alpha_{\mu\sigma\nu}}\delta_{\rho}\phantom{}^\lambda-\tensor R{^\alpha_{\rho\sigma\nu}}\delta_{\mu}\phantom{}^\lambda.$$
Contracting $\mu$ and $\alpha$, recalling that $\tensor R{^\mu_{\nu\mu\rho }}=R_{\nu\rho}$ and $\tensor R{^\mu_{\mu\nu\rho}}=0$, we find
$$\tensor R{^\lambda_{\nu\rho\sigma}}-\tensor R{^\lambda_{\sigma\rho\nu}}+\tensor R{^\lambda_{\rho\sigma\nu}}-N\tensor R{^\lambda_{\rho\sigma\nu}}=-R_{\nu\rho}\delta_\sigma\phantom{}^\lambda+R_{\sigma\rho}\delta_\nu\phantom{}^\lambda-R^\lambda_{\rho\sigma\nu},$$
here we recognize that the algebraic Bianchi identity, $\tensor R{^\lambda_{\sigma\rho\nu}}=-\tensor R{^\lambda_{\sigma\nu\rho}}=\tensor R{^\lambda_{\nu\rho\sigma}}+\tensor R{^\lambda_{\rho\sigma\nu}},$ cancels two terms in the previous equation, giving, after having lowered one index, 
\begin{equation}\label{RN1}
    (N-1)R_{\lambda\rho\sigma\nu}=R_{\nu\rho}g_{\sigma\lambda}-R_{\sigma\rho}g_{\nu\lambda}.
\end{equation}
Notice that the above equation must be antisymmetric in $\lambda\ \rho$ (due to the proprieties of the Riemann tensor), therefore$$
R_{\nu\rho}g_{\sigma\lambda}-R_{\sigma\rho}g_{\nu\lambda}=-R_{\nu\lambda}g_{\sigma\rho}+R_{\sigma\lambda}g_{\nu\rho}.
$$ To conclude, contracting $\lambda\ \nu$, the above gives 
\begin{equation}
    R_{\sigma\rho}-NR_{\sigma\rho}=-Rg_{\sigma\rho}+R_{\sigma\rho},\quad \Rightarrow\quad \boxed{R_{\sigma\rho}=\frac{R}{N}g_{\sigma\rho}},
\end{equation}
and inserting this one into the \eqref{RN1} we get our final result
\begin{equation}
        R_{\lambda\rho\sigma\nu}=\frac{R}{N(N-1)}(g_{\nu\rho}g_{\lambda\sigma}-g_{\sigma\rho}g_{\lambda\nu}).
\end{equation}
As we can see, both the Riemann and Ricci tensor are therefore completely determined by the metric tensor and the Ricci scalar. 