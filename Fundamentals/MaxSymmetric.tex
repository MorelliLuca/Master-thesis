\section{Maximally symmetric spaces}\label{app:MaxSymm}
Consider $\mathbb{R}^n$, this space is highly symmetric: it is isotropic and homogeneous, or, in a simpler way, every point and every direction "look" the same.\\
This means that $\mathbb{R}^n$ is symmetric under every rotation and translation: in $n$-dimensions there are $n$ possible translations (along the $n$ axes) and $n\frac{n-1}{2}$ possible rotations (for each axis we can rotate it towards $n-1$ other axes and to avoid double counting $x\rightarrow y$ and $y\rightarrow x$ we divide by $2$), for a total number of symmetries equals to $$n+n\frac{n-1}{2}=n\frac{n+1}{2}.$$ 
An $n$-dimensional manifold is said to be \textbf{maximally symmetric} if it possesses the same number of symmetries of $\mathbb{R}^n$. In the differential geometry language, a symmetry is defined through isometries, that are diffeomorphisms under which the metric tensor is invariant.\\
For each symmetry of the metric we can define a \textbf{Killing vector}, which satisfies the Killing equation 
\begin{equation}
    0=(\pounds_{\vec K}g)_{\mu\nu}=\nabla_\mu K_\nu+\nabla_\nu K_\mu, \label{KillingEq}
\end{equation}
where $\pounds_{\vec{K}}$ is the Lie derivative along $\vec K$, which is the Killing vector.

We now want to show that a maximally symmetric space really possesses the maximum number of symmetries, namely the maximum number of independent\footnote{Linearly independent here means that $\not\exists$ a set of constants $c_n$ such that $$\sum_n c_n K_\mu^{(n)}(P)=0 \qquad \forall P\in\mathcal{M}.$$} Killing vectors.
Consider the defining equation of the Riemann tensor applied to a 1-form
\begin{equation}
    R^{\mu}_{\nu\rho\sigma}K_\mu=-[\nabla_\rho,\nabla_\sigma]K_\nu,\label{RiemannDef}
\end{equation}
this definition, combined with the algebraic Bianchi identity, ($R^\mu_{\nu\rho\sigma}+R^\mu_{\rho\sigma\nu}+R^\mu_{\sigma\nu\rho}=0 $) implies that each Killing vector must satisfy
$$\nabla_\rho\nabla_\sigma K_\nu-\nabla_\sigma\nabla_\rho K_\nu +\nabla_\sigma\nabla_\nu K_\rho-\nabla_\nu\nabla_\sigma K_\rho+\nabla_\nu\nabla_\rho K_\sigma-\nabla_\rho\nabla_\nu K_\sigma=0.$$
This equation can be simplified by the Killing equation \eqref{KillingEq}: using this relation we can sum pairs of terms obtaining $$ 2(\nabla_\rho\nabla_\sigma K_\nu-\nabla_\sigma\nabla_\rho K_\nu -\nabla_\nu\nabla_\sigma K_\rho)=0,$$ that using \eqref{RiemannDef} turns out to be the following
\begin{equation}
    R^\mu_{\nu\rho\sigma}K_\mu=\nabla_\nu\nabla_\sigma K_\rho. \label{RNabla}
\end{equation}
This equation shows that the second covariant derivative acts on Killing vectors just as a linear application. In this way we can determine every derivative of a Killing vector in a specific point, just by knowing its value and the value of its first covariant derivative, at the same point.\\
If we now Taylor expand the Killing vector around a point $P$, we will obtain some kind of expansion that depends on the value in $P$ of all covariant derivatives of all orders, however we showed that we can evaluate those just knowing $K_\mu(P)$ and $\nabla_\nu K_\mu(P)$. This means that we can express the Killing vector field as a combination of two functions that do not depend on the Killing vector itself or on its derivatives:
\begin{equation*}
    K_\mu(x)=A_\mu\phantom{}^\lambda(x,P)K_\lambda(P)+B_\mu\phantom{}^{\lambda\nu}(x,P)\nabla_\nu K_\lambda(P),
\end{equation*}
these functions depend only on $x$, the point $P$, and the metric, through the Riemann tensor. For this reason these must be the same functions for all Killing vectors:
\begin{equation}\label{KillingExp}
    K_\mu^{(n)}(x)=A_\mu\phantom{}^\lambda(x,P)K_\lambda^{(n)}(P)+B_\mu\phantom{}^{\lambda\nu}(x,P)\nabla_\nu K_\lambda^{(n)}(P).
\end{equation}
The above equation tells us that a given Killing vector is determined by $K_\lambda^{(n)}(P)$, which has $N$ possible independent values, and by $\nabla_\nu K_\lambda^{(n)}(P)$, which has $N\frac{N-1}{2}$ independent values, due to its antisymmetry (which is a consequence of the Killing equation \eqref{KillingEq}).\\
In this way we have shown that the maximum number of independent Killing vectors in an $N$-dimensional manifold is exactly the same number that possesses $\mathbb{R}^N$ $$N+N\frac{N-1}{2}=N\frac{N+1}{2}.$$

We want to conclude deriving the form that has the Riemann tensor in a maximally symmetric space.\\
In general, equation \eqref{RNabla} must hold for every Killing vector, furthermore it also must be consistent with the commutator of covariant derivatives \eqref{RiemannDef}. This requirement and the fact that we have the maximum number of linearly independent Killing vectors will determine the form of $R^\mu_{\nu\rho\sigma}$. Consider \eqref{RiemannDef} applied to the two indices tensor $$ [\nabla_\sigma,\nabla_\nu]\nabla_\mu K_\rho=-R^\lambda_{\mu\sigma\nu}\nabla_\lambda K_\rho-R^\lambda _{\rho\sigma\nu}\nabla_\mu K_\lambda,$$ the equation \eqref{RNabla} can be used to obtain 
\begin{align*}
    \nabla&_\sigma(R^\lambda_{\nu\rho\mu}K_\lambda)-\nabla_\nu(R^\lambda_{\sigma\rho\mu}K_\lambda)=\\&=\nabla_\sigma R^\lambda_{\nu\rho\mu}K_\lambda-\nabla_\nu R^\lambda_{\sigma\rho\mu}K_\lambda+\ R^\lambda_{\nu\rho\mu}\nabla_\sigma K_\lambda- R^\lambda_{\sigma\rho\mu}\nabla_\nu K_\lambda=-R^\lambda_{\mu\sigma\nu}\nabla_\lambda K_\rho-R^\lambda _{\rho\sigma\nu}\nabla_\mu K_\lambda.
\end{align*}
Now, Killing equation \eqref{KillingEq} allows us to move the index $\lambda$ to the covariant derivative in each term, then, using a bunch of Kronecker deltas we get $$ (\nabla_\sigma R^\lambda_{\nu\rho\mu}-\nabla_\nu R^\lambda_{\sigma\rho\mu})K_\lambda=(R^\lambda_{\nu\rho\mu}\delta_{\sigma}\phantom{}^\alpha-R^\lambda_{\sigma\rho\mu}\delta_{\nu}\phantom{}^\alpha+R^\lambda_{\mu\sigma\nu}\delta_{\rho}\phantom{}^\alpha-R^\lambda_{\rho\sigma\nu}\delta_{\mu}\phantom{}^\alpha)\nabla_\lambda K_\alpha.$$
This relation must hold for every Killing vector. We have the maximum number of independent Killing vectors, thus we can generate any other Killing vector from a combination of these. The general expansion \eqref{KillingExp} shows that a Killing vector field that vanishes in $P$, while its derivatives does not, can exists, and we surely can obtain it from a linear combination of the others. The above equation holds also for this one in $P$ only if the right-hand side vanishes too, this can happen only if the term in parentheses is symmetric in $\lambda\ \alpha$ (so that it vanishes when contracted with $\nabla_\lambda K_\alpha$ that is antisymmetric)$$R^\lambda_{\nu\rho\mu}\delta_{\sigma}\phantom{}^\alpha-R^\lambda_{\sigma\rho\mu}\delta_{\nu}\phantom{}^\alpha+R^\lambda_{\mu\sigma\nu}\delta_{\rho}\phantom{}^\alpha-R^\lambda_{\rho\sigma\nu}\delta_{\mu}\phantom{}^\alpha=R^\alpha_{\nu\rho\mu}\delta_{\sigma}\phantom{}^\lambda-R^\alpha_{\sigma\rho\mu}\delta_{\nu}\phantom{}^\lambda+R^\alpha_{\mu\sigma\nu}\delta_{\rho}\phantom{}^\lambda-R^\alpha_{\rho\sigma\nu}\delta_{\mu}\phantom{}^\lambda.$$
Contracting $\mu$ and $\alpha$, recalling that $R^\mu_{\nu\mu\rho }=R_{\nu\rho}$ and $R^\mu_{\mu\nu\rho}=0$, we find
$$R^\lambda_{\nu\rho\sigma}-R^\lambda_{\sigma\rho\nu}+R^\lambda_{\rho\sigma\nu}-NR^\lambda_{\rho\sigma\nu}=-R_{\nu\rho}\delta_\sigma\phantom{}^\lambda+R_{\sigma\rho}\delta_\nu\phantom{}^\lambda-R^\lambda_{\rho\sigma\nu},$$
here we recognize that, from the algebraic Bianchi identity, $$R^\lambda_{\sigma\rho\nu}=-R^\lambda_{\sigma\nu\rho}=R^\lambda_{\nu\rho\sigma}+R^\lambda_{\rho\sigma\nu},$$ which cancels two terms in the previous equation, that now reads, after having lowered one index, 
\begin{equation}\label{RN1}
    (N-1)R_{\lambda\rho\sigma\nu}=R_{\nu\rho}g_{\sigma\lambda}-R_{\sigma\rho}g_{\nu\lambda}.
\end{equation}
Notice that the above equation must be antisymmetric in $\lambda\ \rho$ (due to the proprieties of the Riemann tensor),$$
R_{\nu\rho}g_{\sigma\lambda}-R_{\sigma\rho}g_{\nu\lambda}=-R_{\nu\lambda}g_{\sigma\rho}+R_{\sigma\lambda}g_{\nu\rho},
$$ contracting $\lambda\ \nu$, this relation becomes 
\begin{equation}
    R_{\sigma\rho}-NR_{\sigma\rho}=-Rg_{\sigma\rho}+R_{\sigma\rho},\quad \Rightarrow\quad \boxed{R_{\sigma\rho}=\frac{R}{N}g_{\sigma\rho}},
\end{equation}
inserting this one into the \eqref{RN1} we get our final result
\begin{equation}
        R_{\lambda\rho\sigma\nu}=\frac{R}{N(N-1)}(g_{\nu\rho}g_{\lambda\sigma}-g_{\sigma\rho}g_{\lambda\nu}).
\end{equation}