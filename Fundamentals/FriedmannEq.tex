\section{The Friedmann equations}
We now want to determine the dynamics of the parameters appearing in the Robertson Walker \eqref{eq:RWMetric} metric knowing the energy content of the universe. The connection between the metric and the energy is given by the \emph{Einstein field equations}
\begin{equation}\label{eq:EFE}
    R_{\mu\nu}-\frac{1}{2}Rg_{\mu\nu}=8\pi GT_{\mu\nu},
\end{equation}
where it appears the energy-momentum tensor $T^{\mu\nu}$.
\subsection{Cosmic fluids}
The simplest model for the content of the universe is a \emph{perfect fluid} of energy and matter. A perfect fluid, in general, is described by an energy-momentum tensor given by
\begin{equation}\label{eq:PerfectFluidEMTensor}
    T^{\mu\nu}=(\rho+p)U^\mu U^\nu+pg^{\mu\nu},
\end{equation}
where $\rho$ is the energy density of the fluid, $p$ the pressure and $U^\mu$ the $4$-velocity of a particle of the fluid.\\When we described the coordinates appearing in the Robertson Walker metric, we anticipated that those were comoving coordinates with respect to the content of the universe (so that in that reference frame the metric would be manifestly isotropic and homogeneous). In the reference frame associated to those coordinates, the fluid is at rest, thus its energy-momentum tensor takes the form
\begin{equation}
    U^\mu=(1,0,0,0),\quad\Rightarrow\quad
    T_{\mu\nu}=\begin{pmatrix}
        \rho&0&0&0\\
        0&&&\\
        0&&g_{ij}p&\\
        0&&&
    \end{pmatrix},
    \quad T^\mu\phantom{} _\nu=\text{diag}(-\rho,p,p,p)\label{eq:EMTensorCosmicFluid}
\end{equation}
Even before plugging everything in the Einstein equations, we can study the energy conservation of this fluid, which reads
\begin{align}\label{eq:ConservationEnergy}
    0&=\nabla_\mu T^\mu\phantom{} _0\nonumber\\&=\partial_\mu T^\mu\phantom{} _0+\Gamma^\mu_{\mu\lambda}T^\lambda\phantom{} _0-\Gamma^\lambda_{\mu0}T^\mu\phantom{} _\lambda\nonumber\\
    &=\partial_0 T^0\phantom{} _0+\Gamma^\mu_{\mu0}T^0\phantom{} _0-\Gamma^\lambda_{\mu0}T^\mu\phantom{} _\lambda\nonumber\\
    &=-\dot\rho-3\frac{\dot a}{a}\rho-3\frac{\dot a}{a}p\nonumber\\
    &=-\dot\rho-3\frac{\dot a}{a}(\rho+p),
\end{align}
in which we used that $T^\mu\phantom{}_\nu$ is diagonal, and the Christoffel symbols \eqref{eq:RWChristoffel}.

From now, we assume that the fluid follows some simple equation of state, as
\begin{equation}\label{eq:EquationState}
    p=\omega\rho,\qquad\omega=\text{constant}.
\end{equation}
Inserting this into the conservation of energy equation \eqref{eq:ConservationEnergy} we find
\begin{equation*}
    \frac{\dot\rho}{\rho}=-3(1+\omega)\frac{\dot a }{a},
\end{equation*}
that can be solved to obtain how the energy density of the fluid scales as the universe expands:
\begin{align*}
    \int \frac{d\rho}{\rho}=-3(1+\omega)\int \frac{da}{a} \quad\Rightarrow\quad \boxed{\rho=\rho_0a^{-3(1+\omega)}}.
\end{align*}

To better grasp the physics of our construction let's study the evolution of some simple cases of fluids.
\begin{itemize}
    \item \textbf{Dust}: this kind of fluid is defined as a set of collisionless, non-relativistic particles, that therefore will have zero pressure:$$p_d=\omega_d\rho_d=0,\quad\Rightarrow\quad \omega_d=0\quad\Rightarrow\quad\rho_d=\frac{E}{V}=\rho_{0}a^{-3}.$$ We can appreciate how, for dust, the energy density scales with the volume ($V \propto  a^3$), keeping constant the total energy. This sort of fluid can be used to model groups of stars and galaxies, for which the pressure is negligible, compared to the energy density.
    \item \textbf{Radiation}: in this case we want to describe massless particles or ultra-relativistic ones, which can be approximated to be massless. We can obtain an equation of state for this fluid by first observing that the $T^{\mu\nu}$ is traceless for E-M fields $$T^\mu\phantom{}_\mu=F^{\mu\lambda}F_{\mu\lambda}-\frac{1}{4}g^\mu\phantom{}_\mu F^{\lambda\sigma}F_{\lambda\sigma}=0,$$ at the same time the \eqref{eq:EMTensorCosmicFluid} gives that $$T^\mu\phantom{}_\mu=-\rho+3P,\quad\Rightarrow\quad P_r=\frac{1}{3}\rho_r,$$ which implies $\omega_r=\frac{1}{3}$. Therefore, the energy density of radiation scales as$$\rho_r=\rho_0a^{-4},$$ that means that for radiation the total energy is not conserved. We interpret this as the fact that, while the universe expands, radiation gets redshifted.
    \item \textbf{Vacuum or dark energy}: this last type of cosmic fluid is quite a strange one, the equation of state for this fluid is $$p_v=-\rho_v,\quad \Rightarrow\quad \omega_v=-1.$$ This means that the energy density, as well as the pressure, as the universe expands, remains constant. Sometimes this is not considered a content of the universe, and it is referred as the \emph{cosmological constant} $\Lambda$: $$\rho_v=\frac{\Lambda}{8\pi G}.$$
\end{itemize}
Initially, it was thought that the universe could be described just but dust and radiation: a radiation dominated universe that then transitioned into a matter dominated. This was supported by the fact that $\rho_r\propto a^{-4}$ decrease faster than $\rho_d\propto a^{-3}$, as the universe expands.\\ Nowadays, we know that the expansion of the universe is accelerating, and this led to the introduction of the dark energy, to account for this behavior.
\subsection{Firedmann equations}
Now that we characterized the main types of fluids that we can use to model the content of the universe, we can proceed to derive the equations governing the time evolution of the universe.

First we want to simplify a bit Einstein equations \eqref{eq:EFE}: from the trace of both sides of the equation we get
\begin{equation*}
    R-\frac{4}{2}R=8\pi GT\quad\Rightarrow\quad R=-8\pi GT,
\end{equation*}
where $T=T^\mu\phantom{}_\mu$, plugging this result in the field equations \eqref{eq:EFE}, we can remove the Ricci scalar:\begin{equation*}
    R_{\mu\nu}=8\pi G\bigg(T_{\mu\nu}-\frac{1}{2}Tg_{\mu\nu}\bigg).
\end{equation*}
From the Ricci tensor components of the Robertson Walker metric \eqref{eq:RWRicci} and the energy momentum tensor \eqref{eq:EMTensorCosmicFluid} we can obtain two equations:
\begin{itemize}
    \item the $\mu\nu=00$ component leads to
    \begin{align*}
        -3\frac{\ddot a}{a}&=8\pi G\bigg[-\rho-\frac{1}{2}(-\rho+3p)\bigg]\\&=4\pi G(\rho+3p);
    \end{align*}
    \item the $\mu\nu=ij$ components lead to
    \begin{align*}
        \frac{a\ddot a+2\dot a^2+2k}{a^2}g_{ij}&=8\pi G\bigg[pg_{ij}-\frac{1}{2}g_{ij}(-\rho+3p)\bigg]\\&=4\pi G(\rho-p)g_{ij}.
    \end{align*}
\end{itemize}
Substituting the former into the latter we find
\begin{align}
   -\frac{4}{3}\pi G(\rho+3p) +\frac{2\dot a^2+2k}{a^2}&=4\pi G(\rho-p)\nonumber\\\frac{2\dot a^2+2k}{a^2}&=4\pi G\frac{4}{3}\rho\nonumber
\end{align}
\begin{equation}
    \boxed{\bigg(\frac{\dot a }{a}\bigg)^2=\frac{8\pi G}{3}\rho-\frac{k}{a^2}}\label{eq:Friedmann1},
\end{equation}
which is the \textbf{first Friedmann equation}, while from the $00$ component alone we get the \textbf{second Friedmann equation}
\begin{equation}
    \label{eq:Friedmann2}\boxed{\frac{\ddot a}{a}=-\frac{4\pi G}{3}(\rho+3p)}.
\end{equation}
The first, which is the one that is usually referred as the Friedmann equation, will determine the time evolution of the scale factor $a(t)$. To solve it, it is enough to know the dependence $\rho(a)$, that we previously discussed.
\subsection{Universe geometry and its density}
Usually, the first Friedmann equation \eqref{eq:Friedmann1} is expressed in terms of some cosmological parameters:
\begin{itemize}
    \item the \textbf{Hubble parameter}, $H=\frac{\dot a}{a}$, which measure the rate of expansion,
    \item the \textbf{critical density}, $\rho_{\text{crit}}=\frac{3H^2}{8\pi G}$
    \item the \textbf{density parameter}, $\Omega=\frac{8\pi G}{3H^2}\rho=\frac{\rho}{\rho_{\text{crit}}}$.
\end{itemize}
In this way \eqref{eq:Friedmann1} explicitly relates the matter content of the universe with its geometry (flat. open or closed). Indeed, inserting the above parameters in \eqref{eq:Friedmann1} it reads
\begin{equation}
    \boxed{\Omega-1=\frac{k}{H^2a^2}},
\end{equation}
from which we can distinguish 3 distinct cases:
\begin{itemize}
    \item $\rho<\rho_{\text{crit}}\quad\Leftrightarrow\quad \Omega<1 \quad\Leftrightarrow\quad k<0 \quad\Leftrightarrow\quad $\emph{open universe},
    \item  $\rho=\rho_{\text{crit}}\quad\Leftrightarrow\quad \Omega=1 \quad\Leftrightarrow\quad k=0 \quad\Leftrightarrow\quad $\emph{flat universe},
    \item  $\rho>\rho_{\text{crit}}\quad\Leftrightarrow\quad \Omega>1 \quad\Leftrightarrow\quad k>0 \quad\Leftrightarrow\quad $\emph{closed universe}.
\end{itemize} 
Observations suggest that now, for our universe, $k\approx0$. Therefore, we will always consider flat geometry. In this case the dynamics of the universe is determined by $$\bigg(\frac{\dot a }{a}\bigg)^2=\frac{8\pi G}{3}\rho. $$\pagebreak
Depending on the different cosmic fluids the universe evolves mainly in three different ways.
\begin{itemize}
    \item \textbf{Matter dominated universe}: in this case, the universe is approximated to contain only dust, therefore $\rho=\rho_0a^{-3}$. Plugging this energy density into the above differential equation we get$$\dot a=H_0a^{-\frac{1}{2}}\quad \Rightarrow\quad a(t)=\bigg(\frac{3}{2}H_0t\bigg)^{2/3},$$ where we introduced $H_0=H(t_0)=\sqrt{\frac{8\pi G}{3}\rho_0}$ and imposed $a(0)=0$.\\ This kind of universe is expanding but at a slower and slower rate ($\ddot a\leq0$).
    \item  \textbf{Radiation dominated universe}: now the universe is approximately filled only by radiation, therefore $\rho=\rho_0a^{-4}$. The above differential equation now reads$$\dot a=H_0a^{-1}\quad \Rightarrow\quad a(t)=\sqrt{2H_0t},$$ where again $H_0=H(t_0)=\sqrt{\frac{8\pi G}{3}}$ and we imposed $a(0)=0$.\\ Again, this universe is expanding at a slower and slower rate ($\ddot a\leq0$).
    \item \textbf{Empty universe}: lastly we consider an empty universe or in which vacuum energy dominates, therefore $\rho=\frac{\Lambda}{8\pi G}$, from which we get$$\dot a=a\sqrt{\frac{\Lambda}{3}}\quad \Rightarrow\quad a(t)=a_0e^{\sqrt{\frac{\Lambda}{3}}(t-t_0)},$$
     in which we imposed $a(t_0)=a_0$.\\Note that, among the cases, this universe is the only one that has an accelerating expansion ($\ddot a\geq0$). It is worth noting also that the first two cases admit a finite time (in our calculations $t=0$) for which the universe has no spatial extension ($a(0)=0$ generates a singularity).\\ The empty universe does not admit it. 
\end{itemize}
\subsection{Matter-radiation universe}
asd
\subsection{The $\Lambda$CDM model} 
asd