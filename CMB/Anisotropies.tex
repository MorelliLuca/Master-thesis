\section{Angular power spectrum}\label{sec:AngularPowerSpectrum}
When we observe the \emph{CMB} in the sky, we are interested to measure the temperature of photons coming from a specific direction $\mathbf{\hat{n}}$ to us. In general, this temperature is not perfectly uniform from all directions, we describe these anisotropies in the following way:
$$T(\mathbf{\hat{n}}) = \bar{T} \left[1 + \Theta(\mathbf{\hat{n}})\right] \qquad \text{with } \Theta(\mathbf{\hat{n}}) \stackrel{\text{def}}{=} \frac{\delta T(\mathbf{\hat{n}})}{\bar{T}}$$
and where $\bar{T}$ is the average temperature in the sky and $\delta T(\mathbf{\hat{n}})$ is the temperature fluctuation in the direction $\mathbf{\hat{n}}$.
To compare the temperature at two district points in the sky we define the \emph{two point correlation function}:
\begin{equation} \label{eq:ani2pcf}
    C(\mathbf{\hat{n}},\mathbf{\hat{n}}') \defeq \langle \Theta(\mathbf{\hat{n}})\Theta(\mathbf{\hat{n}}') \rangle,
\end{equation}
here the angle brackets denote an average over an ensemble of universes (It will be discussed later in this section how we can approximate this averaging process).


The most appropriate way to describe the temperature fluctuations, given that these are observed from the sky, is to expand $\Theta$ in spherical harmonics
\begin{equation}\label{eq:harmexpansion}
    \Theta(\mathbf{\hat{n}}) = \sum_{\ell=0}^{\infty} \sum_{m=-\ell}^{\ell} a_{\ell m} Y_{\ell m}(\mathbf{\hat{n}}),
\end{equation}
where the coefficients $a_{\ell m}$, also called \textbf{multipole moments}, are given by
\begin{equation*}
    a_{\ell m} = \int d\Omega\ \Theta(\mathbf{\hat{n}}) Y^*_{\ell m}(\mathbf{\hat{n}}).
\end{equation*}
Also, for the multipole moments we can define a two point correlation function
\begin{equation}\label{eq:angularpowerspectrum}
    \langle a_{\ell m} a^*_{\ell' m'} \rangle = \delta_{\ell \ell'} \delta_{m m'}\ C_{\ell},
\end{equation}
where $C_{\ell}$ is the \textbf{angular power spectrum} and again the angle brackets represent an ensemble average. Sometimes it is also used $\mathcal{D}_\ell \defeq \frac{\ell(\ell+1)}{2\pi}\bar{T}^2C_{\ell}$.

Note that combining \eqref{eq:harmexpansion} and \eqref{eq:ani2pcf} we obtain
\begin{align*}
    C(\mathbf{\hat{n}},\mathbf{\hat{n}}') &= \langle \Theta(\mathbf{\hat{n}})\Theta(\mathbf{\hat{n}}') \rangle \\ &= \sum_{\ell=0}^{\infty} \sum_{m=-\ell}^{\ell}  \sum_{\ell'=0}^{\infty} \sum_{m'=-\ell'}^{\ell'} \langle a_{\ell m} a_{\ell' m'} \rangle Y_{\ell m}(\mathbf{\hat{n}})Y_{\ell' m'}(\mathbf{\hat{n}}')\\
    &= \sum_{\ell=0}^{\infty}C_{\ell} \sum_{m=-\ell}^{\ell}  Y_{\ell m}(\mathbf{\hat{n}})Y^*_{\ell m}(\mathbf{\hat{n}}')\\
    &\ \Bigg\downarrow\quad \text{using } \sum_{m=-\ell}^{\ell}Y_{\ell m}(\mathbf{\hat{n}})Y^*_{\ell m}(\mathbf{\hat{n}}')=\frac{2\ell+1}{4\pi}P_{\ell}(\cos\theta)\\
    &=\sum_{\ell=0}^{\infty} C_{\ell}\frac{2\ell+1}{4\pi}P_{\ell}(\cos\theta),
    \end{align*}
where $P_{\ell}$ are the Legendre polynomials and $\theta$ is the angle between $\mathbf{\hat{n}}$ and $\mathbf{\hat{n}}'$.
Invoking the orthogonality of the Legendre polynomials we can write
\begin{equation}\label{eq:cl}
    C_{\ell} = 2\pi\int_{-1}^{1}d(\cos\theta)\ P_{\ell}(\cos\theta)C(\mathbf{\hat{n}},\mathbf{\hat{n}}'),
\end{equation}
this shows that the angular power spectrum encodes the same information as the two point correlation function \eqref{eq:ani2pcf}, hence it measures the correlation between the temperature fluctuations at two points in the sky separated by an angle $\theta$.

\subsection{Estimate the angular power spectrum}\label{sec:EstimateCl}
We now want to understand how can we estimate the average over the ensemble of universes in the previous definitions. Note that, fixed $\ell$, we still get $2\ell+1$ different values of $a_{\ell  m}$, this allows us to estimate the angular power spectrum as
\begin{equation}\label{eq:cl_estimate}
    \hat{C}_{\ell} = \frac{1}{2\ell+1}\sum_{m=-\ell}^{\ell} |a_{\ell m}|^2.
\end{equation}
One can show that this estimator is unbiased\footnote{An estimator is said to be unbiased if its expected value is equal to the true value of the parameter being estimated.}, however its variance is non-zero:
\begin{equation}\label{eq:cl_variance}
    \Delta\hat{C}_{\ell} \defeq\sqrt{\langle(C_\ell -\hat C_\ell)^2\rangle} = \sqrt{\frac{2}{2\ell+1}}C_{\ell},
\end{equation} 
this error that systematically appears in this estimate is usually called \textbf{cosmic variance}. Cosmic variance will result in a larger error for smaller values of $\ell$, which corresponds to larger angular scales. This can be understood as a consequence of the fewer number of modes $a_{\ell m}$ available at lower $\ell$.

\subsection{Multipole expansion}\label{sec:MultipoleExpansion}
In the previous section we considered the temperature fluctuations observed by us in the sky, therefore it was natural to assume that these were functions of the direction of observation $\versor n$.\\In general however, we should consider that these anisotropies varies also with the position of the observer in spacetime. This broader view is needed since to predict the observations we will need to describe the evolution of the anisotropies throughout the whole universe. Therefore, we will now consider
\begin{equation}
    \Theta({t,\mathbf x,\versor p})\quad\text{with}
    \begin{cases}
        t\quad\text{cosmic time,}\\
        \mathbf x\quad\text{position of the anisotropy in space,}\\
        \versor p\quad\text{direction of motion of the photons}.
    \end{cases}
\end{equation}
To come back to the observed anisotropies we just fix $t$ at the present day, $\mathbf x$ on the earth and we consider the direction of motion of the photons as the direction of observation (since it is the direction from which they come from).\\

In section \ref{sec:ThetaTimeEvolution} we will see that the evolution of the anisotropies is described by a linear differential equation (since we are working with first order perturbations). It is therefore useful to introduce here some expansions that will simplify these equations.

First of all, we can simplify the spacial dependence moving to Fourier space
\begin{equation}\label{eq:fourier_expansion}
    \Theta(t,\mathbf x,\versor p) = \int \frac{d^3k}{(2\pi)^3}e^{i\mathbf{k}\cdot\mathbf{x}}\tilde\Theta(\mathbf{k},t,\versor p),
\end{equation} 
where $\tilde\Theta$ is the Fourier transform of $\Theta$.
In this way, we obtained a decomposition on plane waves that leaves $\tilde\Theta$ depending on two vectors, $\mathbf k$ and $\versor p$. However, since the background spacetime is homogeneous and isotropic, the real useful information is encoded in one of these two vectors and in the angle between them. This allows us to define $$\mu=\frac{\mathbf{k}\cdot \versor p}{k}\qquad\Rightarrow\qquad \tilde\Theta(t,\mathbf k,\mu)\quad \text{with } \mu\in[-1,1].$$
This suggests us to that another useful expansion is the \textbf{Legendre polynomial expansion}:
\begin{align}\label{eq:legendre_expansion}
    &\tilde\Theta(t,\mathbf k,\mu) = \sum_{\ell=0}^{\infty}\frac{2\ell+1}{i^\ell} \tilde\Theta_{\ell}(t,\mathbf{k})P_{\ell}(\mu),\\
    &\tilde\Theta_{\ell}(t,\mathbf{k}) =\frac{i^\ell}{2} \int_{-1}^{1}d\mu\ P_{\ell}(\mu)\tilde\Theta(t,\mathbf k,\mu)\nonumber,
\end{align} 
where $P_{\ell}$ are the Legendre polynomials and $\tilde\Theta_\ell$ are the \textbf{multipoles}.\\
The Legendre polynomials can be computed recursively using the Bonnet's formula
\begin{equation}\label{eq:bonnet}
    (\ell+1)P_{\ell+1}(\mu) = (2\ell+1)\mu P_{\ell}(\mu)-\ell P_{\ell-1}(\mu),
\end{equation}
and knowing that $P_0=1$, $P_1=\mu$ and $P_2=\frac{3\mu^2-1}{2}$.
\subsection{From perturbations to anisotropies}\label{sec:PertToAnis}
It is now time to discuss how in general we connect the perturbations of the FRW universe to the power spectrum of the anisotropies that we observe in the \emph{CMB}. The following sections will be devoted to understand how to get all the physical quantities that are needed to relate perturbations to the anisotropies.\\

We are interested in evaluating $\langle\tilde\Theta(\mathbf{k},\versor n)\tilde\Theta(\mathbf{k}',\versor n')\rangle$. This quantity is determined by two phenomena:
\begin{enumerate}
    \item the initial amplitude of the perturbations generated during inflation, which from our point of view are random variables generated by vacuum fluctuations;
    \item the evolution of these perturbations that turns these perturbations into the anisotropies that we observe today, this process is clearly deterministic. 
\end{enumerate}
This consideration allows us to proceed in the following way: considering the initial perturbations $\mathcal{R}(\mathbf{k})$ we can decompose $ \tilde\Theta(\mathbf{k},\versor n)=\mathcal{R} (\tilde\Theta/\mathcal{R}) $, now the ratio $\tilde\Theta/\mathcal{R}$ is completely independent of the initial amplitude of the perturbation and won't contribute to the ensemble average.\\
In this way we get
\begin{align*}
    \langle\tilde\Theta(\mathbf{k},\versor n)\tilde\Theta^*(\mathbf{k}',\versor n')\rangle &= \langle\mathcal{R}(\mathbf{k})\mathcal{R}(\mathbf{k}')\rangle\frac{\tilde\Theta(\mathbf{k},\versor n)}{\mathcal{R}(\mathbf{k})}\frac{\tilde\Theta^*(\mathbf{k}',\versor n')}{\mathcal{R}^*(\mathbf{k}')}\\
    &= (2\pi)^3\delta^{(3)}(\mathbf{k}-\mathbf{k}')\mathcal P_{\mathcal{R}}(k)\frac{\tilde\Theta(\mathbf{k},\versor n)}{\mathcal{R}(\mathbf{k})}\frac{\tilde\Theta^*(\mathbf{k}',\versor n')}{\mathcal{R}^*(\mathbf{k}')},
\end{align*}
where we used the definition of the perturbation power spectrum. In this expression the last two factors now depend only on the magnitude of $\mathbf k$ and $\mathbf k'$\\Now, by inserting this result in the expression for the $C_\ell$ \eqref{eq:cl} we find 
\begin{align}
    C_{\ell}&=2\pi\int_{-1}^{1}d\mu P_{\ell}(\mu)\int\frac{d^3k}{(2\pi)^3}\int\frac{d^3k}{(2\pi)^3}e^{i(\mathbf k-\mathbf k')\cdot \mathbf x}\langle\tilde\Theta(\mathbf k,\versor n)\tilde\Theta^*(\mathbf k',\versor n')\rangle\nonumber\\
    &=2\pi\int\frac{d^3k}{(2\pi)^3}\mathcal P_{\mathcal{R}}(k)\int_{-1}^{1}d\mu P_{\ell}(\mu)\frac{\tilde\Theta(\mathbf{k},\versor n)}{\mathcal{R}(\mathbf{k})}\frac{\tilde\Theta^*(\mathbf{k},\versor n')}{\mathcal{R}^*(\mathbf{k})}\nonumber\\
    &=2\pi\int\frac{dk\ k^2}{(2\pi)^3}\mathcal P_{\mathcal{R}}(k)\int_{-1}^{1}d\mu P_{\ell}(\mu)\sum_{\ell',\ell''}\frac{\tilde\Theta_{\ell'}}{\mathcal{R}}\frac{\tilde\Theta_{\ell''}^*}{\mathcal{R}^*}(2\ell'+1)(2\ell''+1)i^{\ell'-\ell''}\times\nonumber\\
    &\qquad\qquad\qquad\qquad\qquad\qquad\qquad\times\int_0^{2\pi} d\phi \int_{-1}^{1} d\cos\theta\ P_{\ell'}(\versor n\cdot\versor k)P_{\ell''}(\versor n'\cdot\versor k)\nonumber\\
    &\qquad\bigg\downarrow\text{using}\int_0^{2\pi} d\phi \int_{-1}^{1} d\cos\theta\  P_\ell(\versor{k}\cdot\versor n)P_{\ell'}(\versor{k}'\cdot\versor n')=\frac{4\pi}{2\ell+1}P_\ell(\versor n\cdot\versor n')\delta_{\ell\ell'}\nonumber\\
    &=8\pi^2\int \frac{dk\ k^2}{(2\pi)^3}\mathcal{P}_{\mathcal R}(k)\sum_{\ell'=0}^{\infty}(2\ell'+1)\bigg|\frac{\tilde\Theta_{\ell'}(\mathbf k,\versor n)}{\mathcal{R(\mathbf k)}}\bigg|^2\int_{-1}^{1}d\mu P_{\ell}(\mu)P_{\ell'}(\mu)\nonumber\\&\qquad\bigg\downarrow\text{orthogonality}\int_{-1}^{+1}d\mu\ P_\ell(\mu)P_{\ell'}(\mu)=\frac{2}{2\ell+1}\delta_{\ell\ell'}\nonumber\\
    &=16\pi^2\int \frac{dk\ k^2}{(2\pi)^3}\mathcal{P}_{\mathcal R}(k)\bigg|\frac{\tilde\Theta_\ell(\mathbf k,\versor n)}{\mathcal{R(\mathbf k)}}\bigg|^2=\frac{2}{\pi}\int dk\ k^2\mathcal{P}_{\mathcal R}(k)\bigg|\frac{\tilde\Theta_\ell(\mathbf k,\versor n)}{\mathcal{R(\mathbf k)}}\bigg|^2
    ,\label{eq:cl_pert}
\end{align}
where $\mu=\cos(\versor n\cdot\versor n')$ and we used the orthogonality of the Legendre polynomial and that $\Theta$ is real.\\
Lastly, introducing the dimensionless power spectrum $\Delta^2_{\mathcal R}(k)\defeq\frac{k^3}{2\pi^2}\mathcal P_{\mathcal R}(k)$ we obtain:
\begin{equation}
   \boxed{ C_{\ell}=4\pi\int \frac{dk}{k}\ \Delta^2_{\mathcal R}(k)\bigg|\frac{\tilde\Theta_\ell(\mathbf k,\versor n)}{\mathcal{R(\mathbf k)}}\bigg|^2.}
\end{equation}
We ended with a formula that relates the angular power spectrum to the power spectrum of the perturbations via the so-called \textbf{transfer function} $|\frac{\Theta_\ell(\mathbf k,\versor n)}{\mathcal{R(\mathbf k)}}|$, which describes how the perturbations generates anisotropies and that we have to find in the next sections.\\
\section{Time evolution of anisotropies}\label{sec:ThetaTimeEvolution}
In this section we want to develop the machinery needed to understand how the anisotropies of the \emph{CMB} observed today evolved from the anisotropies at recombination.

To tackle this problem we need to study the evolution of the phase space of photons in perturbed spacetime. Imposing the \emph{newtonian gauge} we can write the metric as
$$ds^2 = -(1+2\Psi)dt^2 + a^2(t)(1-2\Phi)\delta_{ij}dx^idx^j,$$
in appendix \ref{app:scalarPerturbedLiouvilleOperator} we show that the \emph{Liouville operator} can be expressed as \eqref{eq:Liouville_scalar_perturbed}:
\begin{align*}
    \hat{L}[f]=\frac{\partial f}{\partial t}+\frac{\partial f}{\partial x^i}\frac{\hat p^i}{a}-p\bigg(H-\frac{\partial \Phi}{\partial t}+\frac{\partial \Psi}{\partial x^i}\frac{\hat p^i}{a}\bigg)\frac{\partial f}{\partial p},
\end{align*}
where $p^i=\hat p^i p$ is the local 3-momentum.\\In the above $f=f(x^\mu,p^i)$, however we know that at the background level the phase space distribution should depend only on $(t,p)$ (due to homogeneity and isotropy of the universe). For this reason we should also decompose the distribution in:
\begin{equation}\label{eq:phspdist_perturb}
    f(x^\mu,\mathbf p) = \bar f(t,p) + \Upsilon (x^\mu,\mathbf{p}),
\end{equation}
where $\Upsilon$ is the perturbation of the phase space distribution function.\\
We can get an expression for this perturbation considering a \emph{blackbody radiation} distribution with a fluctuating temperature $T(x^\mu,\versor p)=\bar T (1+\Theta(x^\mu,\versor{p}))$.\\ Note that now $\Theta$ depends on the time ($t=x^0$) and position ($x^i$) of observation, other than the direction of motion of the photons, which corresponds to the direction of observation $\versor{n}$ of section \ref{sec:AngularPowerSpectrum}, where the position and time of observation were fixed by the Earth position in spacetime.\\  In this way, expanding in $\Theta$ we find
\begin{align*}
    f(x^\mu,p^i) &= \bigg[\exp\bigg\{\frac{p}{k_B\bar T(1+\Theta)}\bigg\}-1\bigg]^{-1}\\&\approx\frac{1}{e^{\frac{p}{k_B\bar T}}-1}+\frac{e^{\frac{p}{k_B\bar T}}}{(e^{\frac{p}{k_B\bar T}}-1)^2}\frac{p}{k_B \bar T}\Theta=\bar f-\Theta p \frac{\partial\bar f}{\partial p}\\
    \Longrightarrow \Upsilon &= -\Theta p \frac{\partial\bar f}{\partial p}.
\end{align*}
Expanding the distribution function also in the Liouville operator we get, at the first order
\begin{equation}\label{eq:liouville_pert}
    \hat L[\Upsilon] = -p\frac{\partial\bar f}{\partial p}\bigg[\frac{\partial \Theta}{\partial t}+\frac{\hat p^i}{a}\frac{\partial \Theta}{\partial x^i}-\frac{\partial\Psi}{\partial t}+\frac{\hat p^i}{a}\frac{\partial \Phi}{\partial x^i}\bigg],
\end{equation}
where the first two terms describe free streaming (free motion of photons without scatterings) while the last two terms account for the effect of gravity.\\

To complete the Boltzmann equation we need to consider the first order collision term describing Compton scatterings (as explained by Dodelson \cite{dodelson}):
\begin{equation} \label{eq:first_collision_term}
    C[\Upsilon]|_{\text{\textbf{CS}}} = -p\frac{\partial\bar f}{\partial p} n_{e}\sigma_T\bigg[\Theta_0-\Theta+\mathbf{\hat{p}}\cdot\mathbf{v}_{b}\bigg],
\end{equation}
where $\mathbf{v}_{b}$ is the \textbf{electron bulk velocity} and $\Theta_0$ is the \textbf{anisotropy monopole}, defined as 
$$\Theta_0(x^\mu)=\frac{1}{4\pi}\int d\Omega_{\versor{p}} \Theta(x^\mu,\versor{p}).$$
Let's appreciate that the collision term, assuming $\mathbf{v}_b=0$, will vanish, and thus give equilibrium, if the anisotropies $\Theta(\versor{p})=\Theta_0$.\\
Equating the Liouville operator \eqref{eq:liouville_pert} with the collision term \eqref{eq:first_collision_term} we obtain
\begin{equation}\label{eq:phot_boltzmann_pert}
    \frac{\partial \Theta}{\partial t}+\frac{\hat p^i}{a}\frac{\partial \Theta}{\partial x^i}-\frac{\partial\Psi}{\partial t}+\frac{\hat p^i}{a}\frac{\partial \Phi}{\partial x^i}=n_{e}\sigma_T\bigg[\Theta_0-\Theta+\mathbf{\hat{p}}\cdot\mathbf{v}_{b}\bigg]
\end{equation}
which is the equation describing the dynamics of the CMB anisotropies.\\
Since this equation is a linear partial differential equation, it can be reduced to an ordinary differential equation by Fourier transforming the spatial coordinates.\\
Introducing 
$$\Theta(x^\mu) = \int \frac{d^3k}{(2\pi)^3}e^{i\mathbf{k}\cdot\mathbf{x}}\tilde\Theta(\mathbf{k},t),\qquad \mu\defeq \cos\theta=\frac{\mathbf{k}\cdot\mathbf{\hat{p}}}{k},$$respectively the Fourier transform of $\Theta$ and the cosine of the angle between $\mathbf{k}$ and $\mathbf{\hat{p}}$, and assuming that $\mathbf{v}_b$ is irrotational ($\mathbf{\tilde v}_b=\versor k\tilde v_b$) we can write the Boltzmann equation as
\begin{equation*}
    \frac{\partial \tilde\Theta}{\partial t} +\frac{ik\mu}{a}\tilde\Theta+\frac{\partial \tilde\Psi}{\partial t}+\frac{ik\mu}{a}\tilde\Phi=n_e \sigma_T\Bigg[\tilde\Theta_0-\tilde\Theta+\mu\tilde v_b\Bigg].
\end{equation*}
If we were to account also for the angular dependence of Compton scatterings (as explained in \cite{dodelson}) we would have obtained:
\begin{equation}
    \frac{\partial \tilde\Theta}{\partial t} +\frac{ik\mu}{a}\tilde\Theta+\frac{\partial \tilde\Psi}{\partial t}+\frac{ik\mu}{a}\tilde\Phi=n_e \sigma_T\Bigg[\tilde\Theta_0-\tilde\Theta+\mu\tilde v_b-\frac{3\mu^2-1}{4}\tilde\Theta_2\Bigg],
\end{equation}
where $\tilde\Theta_2\defeq -\frac{1}{2}\int_{-1}^{+1}d\mu\frac{3\mu^2-1}{2}\tilde\Theta$ is the \textbf{anisotropy quadrupole}.
\subsection{Polarization of light}\label{sec:PhotonsPolarization}
In the previous section we studied how the phase space of photons evolve in a perturbed spacetime. However, we have not yet considered that photons are spin 1 particles, and thus, to fully describe them, we also need to know their polarization.

To better understand how polarization works, let's consider a monochromatic plane wave (which we could consider as a Fourier component of a generic wave). The electric and magnetic fields of such a wave, in empty space, are not independent, due to Maxwell equations, and thus we can just focus on the electric field.\\ 
If the wave is propagating along the $\versor z$ axis, its electric field can be written as
$$ \mathbf{E}(z,t) = \text{Re}\bigg\{(E_{x}\versor x+E_y\versor y)e^{ik(z-t)}\bigg\},$$
where $E_x$ and $E_y$ are the components of the electric field in complex space. Since they are complex number we can decompose them in $E_x=|E_x|e^{i\phi_x},\ E_y=|E_y|e^{i\phi_y}$, in this way the monochromatic wave reads:
$$ \mathbf{E}(z,t) = |E_x|\cos[k(z-t)]\versor x+|E_y|\cos[k(z-t)+\phi]\versor y \qquad\text{with}\ \phi=\phi_y-\phi_x.$$
This shows that the electric field, at a fixed $z=z_0$, evolves drawing an ellipse in the $xy$ plane. Note that this ellipse can degenerate depending on the values of $|E_x|$, $|E_y|$ and $\phi$:
\begin{itemize}
    \item if $\phi=0,\pi$ or if one of the components $E_x$, $E_y$ vanishes, the ellipse degenerates into a line, we call this case \textbf{linear polarization};
    \item if $\phi=\pm\frac{\pi}{2}$ and $E_x=E_y$, the ellipse degenerates into a circle, we call this case \textbf{circular polarization}.
\end{itemize}
In general, we can describe the state of a photon by the \textbf{Stokes parameters} 
\begin{align}\label{eq:stokes_parameters}
    I&\defeq |E_x|^2+|E_y|^2,\qquad &Q\defeq |E_x|^2-|E_y|^2,&&\nonumber\\ U&\defeq 2|E_x||E_y|\cos\phi,\qquad &V\defeq 2|E_x||E_y|\sin\phi,&&
\end{align}
where $I$ is the intensity of the light while $Q,U$ and $V$ describe the polarization. These last three parameters could also be interpreted as the difference of the intensity of the electric field components along different orthogonal axis.\\Circular polarization is not produced in the early universe, therefore we will set $V=0$, so that we are describing only linearly polarized or unpolarized light.

Before proceeding, we should note that under rotations in the $xy$ plane $$E_x\rightarrow E_x\cos\theta-E_y\sin\theta,\qquad E_y\rightarrow E_x\sin\theta+E_y\cos\theta,$$ the  Stokes parameters will transform as
$$I\rightarrow I,\qquad Q\pm iU\rightarrow e^{\pm 2i\theta}(Q\pm iU).$$
This transformation shows that the combination $(Q\pm iU)$ transforms as a \emph{spin-2 tensor} while $I$ as a scalar. This observation will be crucial when will need to decompose these modes.
\subsection{Polarization from Compton scattering}\label{sec:ComptonPolarization}

The main mechanism that influence the evolution of CMB is \emph{Compton scattering}. This process can also induce polarization in the photons: indeed, if we consider an electron, on which light can be scatter off, the interaction can absorb some components of the electric field modifying the polarization of the photon.

For example, an unpolarized photon moving along the $x$ axis and deflected along the $z$ axis, in the end, will have a polarization along the $y$ axis. This is due to the simple fact that $\mathbf{E}$ and $\mathbf{B}$ must be orthogonal to the direction of motion and therefore any component along the $z$ axis will be absorbed by the electron.

Consider now some incoming radiation with polarization $\boldsymbol{\epsilon}_i'$\footnote{$\boldsymbol{\epsilon}_i'$ are the versos onto which the $\mathbf{E}$ decomposes.} which gets scattered off by an electron. The deflected radiation will instead have a polarization $\boldsymbol{\epsilon}_i$. Without loss of generality, we can orient our coordinate axis such that the outgoing radiation is travelling along the $z$ axis and the polarization $\boldsymbol{\epsilon}_1=\versor x$ and $\boldsymbol{\epsilon}_2=\versor y$. 

The parameter $Q$, after the scattering, can be estimated decomposing the incoming polarization on the outgoing ones and then averaging over all possible incoming photons:
$$Q\propto\int d\Omega_{in} f_{\text{in}}(\versor n')\sum_{i=1}^{2}\bigg[|\boldsymbol{\epsilon}_{i}'\cdot\versor{x}|^2-|\boldsymbol{\epsilon}_{i}'\cdot\versor{y}|^2\bigg],$$
where $f_{\text{in}}$ is the phase space distribution of the incoming photons.\\
As a function of the polar incoming angles, the incoming polarization can be written as
\begin{align*}
    \boldsymbol{\epsilon}_1'(\theta',\phi') &=(\cos\theta'\cos\phi',\cos\theta'\sin\phi',-\sin\theta'),\\
    \boldsymbol{\epsilon}_2'(\theta',\phi') &=(-\sin\phi',\cos\theta',0).
\end{align*} 
Once inserted in the previous integral we find
\begin{align*}
    Q&\propto\int d\Omega_{in} f_{\text{in}}(\versor n')\bigg[\cos^2\theta'\cos^2\phi'+\sin^2\phi'-\cos^2\theta'\sin^2\phi'-\cos^2\phi'\bigg]\\&\propto\int d\Omega_{in} f_{\text{in}}(\versor n')(\sin^2\theta'\cos2\phi')\propto\int d\Omega_{\text{in}}f_{\text{in}}(\versor n')\bigg[Y_{2,2}(\versor n')+Y_{2,-2}(\versor n')\bigg],
\end{align*}
where we recognized, in the last step, that $\sin^2\phi'\cos2\phi'$ is proportional to the sum of two spherical harmonics\footnote{Recall that $Y_{\ell,\pm\ell}(\theta,\phi)=\frac{(\mp)^\ell}{2^\ell\ell!}\sqrt{\frac{(2\ell+1)!}{4\pi}}\sin^\ell\theta\ e^{\pm i\ell\phi}$ and therefore $Y_{2,2}+Y_{2,-2}\propto sin^2\theta\cos2\phi$.}.\\
Now, considering perturbations of the temperature in $f_{\text{in}}$, as in \eqref{eq:phspdist_perturb}, we discover that, since the integral picks the modes with $\ell=2$, polarization will be generated through Compton scatterings by the quadrupole anisotropy $\Theta_2$. Similar calculations can lead to the same conclusion for the parameter $U$.

Following Dodelson \cite{dodelson}, we will derive the Boltzmann equation for the polarization. To begin, we define the polarization anisotropy, $\Theta_P(\versor n, \mathbf{k})$: consider polarized light  with polarization vectors aligned with the $x$ and $y$ axes (thus propagating in the $z$ direction), the stokes parameter $Q$ will therefore measure the difference of the intensity associated with each polarization. We can then associate to each intensity a temperature (recall $\rho\propto T^4$) so that the Stokes parameter $Q$ can be interpreted as a measure of the difference of the temperature associated with the two polarization states. In this way we can define (using $\delta\rho/\rho\propto4\delta T/T $) 
\begin{equation}\label{eq:ThetaP}
    \Theta_P\defeq\frac{Q}{4I}=\frac{\delta T}{T}\bigg|_{\text{Polarization}}.
\end{equation}
To consider a generic polarization state we just need to rotate the coordinate system, in this way we can write the Stokes parameters as
$$\frac{Q}{4I}=\Theta_P\cos2\phi_k,\qquad \frac{U}{4I}=\Theta_P\sin2\phi_k,$$where $\phi_k$ is the angle between $\mathbf{k}$ and the $x$ axis if we assume the motion of the photon along the $z$ axis.

$\Theta_P$ will then evolve with its own Boltzmann equation:
all the physics described in the previous section is unchanged, however we need to account for Compton scattering effect con polarization. Indeed, we already discussed that the quadrupole $\Theta_2$ will polarize scattered photons, therefore a collision term proportional to $\Theta_2$ must be added to the Boltzmann equation \eqref{eq:phot_boltzmann_pert}. However, if polarization is not sourced, through Compton scattering the radiation will gradually become unpolarized. This means that now a term proportional to $-\Theta_P$ must be added as a collision contribution. The final result (Bond and efstathiou 1987) is the Boltzmann equation for the polarization anisotropy:
\begin{equation}\label{eq:ThetaP_Boltzmann}
    \frac{\partial \tilde\Theta_P}{\partial t}+\frac{ik\mu}{a}\tilde\Theta_P=-n_e\sigma_T\bigg[\Theta_{P}+\frac{1}{2}\bigg(1-P_2(\mu)\bigg)\Pi\bigg],
\end{equation}
where $\Pi=\Theta_2+\Theta_{P,2}+\Theta_{P,0}$ and $P_2(\mu)=\frac{3\mu^2-1}{2}$ is the order 2 Legendre polynomial.

As we saw Compton scattering is influenced by the polarization of photons, taking into account its effect also the Boltzmann equation \eqref{eq:phot_boltzmann_pert} must be corrected:
\begin{equation}\label{eq:phot_boltzmann_pert_pol}
    \frac{\partial \tilde\Theta}{\partial t} +\frac{ik\mu}{a}\tilde\Theta+\frac{\partial \tilde\Psi}{\partial t}+\frac{ik\mu}{a}\tilde\Phi=n_e \sigma_T\Bigg[\tilde\Theta_0-\tilde\Theta+\mu\tilde v_b-\frac{1}{4}P_2(\mu)\Pi\Bigg].
\end{equation}
\subsection{Multipole expansion of the Boltzmann equation}\label{sec:BoltzmannMultipoleExpansion}
In section \ref{sec:ComptonPolarization} we obtained the differential equations \eqref{eq:phot_boltzmann_pert_pol} and \eqref{eq:ThetaP_Boltzmann} governing the time evolution of the anisotropies in the CMB. To end our discussion of the time evolution of the anisotropies we want to expand these equations in multipoles.

Since the CMB is observed in the sky, spherical harmonics are the natural basis to use to project the anisotropies. The fact that the equations \eqref{eq:phot_boltzmann_pert_pol} and \eqref{eq:ThetaP_Boltzmann} depend only on $\mu=\versor p\cdot\versor k$ corresponds to a rotational symmetry, of the system, around one of these two vectors. By using spherical polar coordinates, such that the vector $\versor k$ lies on the $z$ axis, the above rotational symmetry corresponds to a rotational symmetry of the azimuthal angle $\phi$. Considering that $Y_{\ell m}\propto e^{im\phi}$, we immediately recognize that such symmetry is respected only by spherical harmonics with $m=0$ and these precisely corresponds to the Legendre polynomials. Therefore, for scalar perturbation, we can limit ourselves to a multipole expansion on the Legendre polynomials without worrying of all the spherical harmonics.\\

By multiplying the \eqref{eq:phot_boltzmann_pert_pol} by the order $\ell$ Legendre polynomial $P_\ell(\mu)$ and integrating over $\mu$ we can exploit the orthogonality of the Legendre polynomials as follows.
\begin{itemize}
    \item $\frac{\partial \tilde\Theta}{\partial t} $ and $n_e\sigma_T\tilde{\Theta}$ depending on $\mu$ in this expansion will give contributions corresponding respectively to $\frac{\partial \tilde\Theta_\ell}{\partial t} $ and $n_e\sigma_T\tilde{\Theta}_\ell$.
    \item $\frac{\partial\tilde\Psi}{\partial t}$ and $n_e\sigma_T\tilde\Theta_0$ have no $\mu$ dependence, which corresponds to the zeroth order Legendre polynomial $P_0(\mu)=1$, and thus they only contribute to the $\ell=0$ equation.
    \item $\tilde\Phi$ and $n_e\sigma_T\tilde v_b$ are multiplied by $P_1(\mu)=\mu$, giving contributions only to $\ell=1$ equation, while $\Pi$ is multiplied by $P_2(\mu)=\frac{3\mu^2-1}{2}$, contributing only to $\ell=2$ equation. Note that these terms must also be multiplied by a factor corresponding to the integral of their respective Legendre polynomial, since they don't contain any $\tilde\Theta$ function to be expanded 
    \item $\frac{ik\mu}{a}\tilde\Theta$ is instead more complicated since it is the product of two functions depending on $\mu$. Bonnet's formula \eqref{eq:bonnet} allows us to simplify the corresponding integral 
    $$\frac{i^\ell}{2}\int_{-1}^{+1}d\mu\ \mu P_\ell(\mu)\tilde\Theta=\frac{i^\ell}{2}\int_{-1}^{+1}d\mu \bigg[\frac{\ell+1}{2\ell+1}P_{\ell+1}(\mu)+\frac{\ell}{2\ell+1}P_{\ell-1}(\mu)\bigg]\tilde\Theta,$$
    in this way this will give contributions to all the equations coupling them together.
\end{itemize}
Putting all of this together we obtain the following coupled system of differential equations
\begin{subequations}\label{eq:multipole_boltzmann_photons}
    \begin{align}
           &\dot{\tilde{\Theta}}_0=-\frac{k}{a}\tilde{\Theta}_1+\dot{\tilde\Psi}\label{eq:multipole_boltzmann_photons_0}\\
            &\dot{\tilde{\Theta}}_1=\frac{k}{3a}\tilde\Theta_0-\frac{2k}{3a}\tilde\Theta_2+\frac{k}{3}\tilde\Phi-n_e\sigma_T\bigg[\tilde\Theta_1+\frac{\tilde v_b}{3}\bigg]\label{eq:multipole_boltzmann_photons_2}\\
            &\dot{\tilde{\Theta}}_\ell=\frac{\ell k}{(2\ell+1)a}\tilde\Theta_{\ell-1}-\frac{(\ell+1)k}{(2\ell+1)a}\tilde\Theta_{\ell+1}-n_e\sigma_T\bigg[\tilde\Theta_\ell-\frac{\delta_{\ell,2}}{10}\Pi\bigg]\qquad \ell\geq 2,\label{eq:multipole_boltzmann_photons_3}
        \end{align}
\end{subequations}

Similarly, equation \eqref{eq:ThetaP_Boltzmann} will result in the following system of differential equations
\begin{subequations}\label{eq:multipole_boltzmann_polatization}
    \begin{align}
           &\dot{\tilde{\Theta}}_{P0}=-\frac{k}{a}\tilde{\Theta}_{P1}-n_e\sigma_T\bigg[\tilde\Theta_{P0}-\frac{1}{2}\Pi\bigg]\label{eq:multipole_boltzmann_polatization_0}\\
            &\dot{\tilde{\Theta}}_{P\ell}=\frac{\ell k}{(2\ell+1)a}\tilde\Theta_{P\ell-1}-\frac{(\ell+1)k}{(2\ell+1)a}\tilde\Theta_{P\ell+1}-n_e\sigma_T\bigg[\tilde\Theta_{P\ell}-\frac{\delta_{\ell,2}}{10}\Pi\bigg]\qquad \ell\geq 1,\label{eq:multipole_boltzmann_polatization_2}.
        \end{align}
\end{subequations}  
Equations \eqref{eq:multipole_boltzmann_photons} and \eqref{eq:multipole_boltzmann_polatization} are not the full system of coupled equations, indeed in these equations depend on the potential $\tilde\Psi$ and $\tilde\Phi$ and on the electron bulk velocity $\tilde v_b$. The differential equations governing these quantities must then be added to the ones above and solved all toghther.
\subsection{Polarization power spectrum}\label{sec:PolarizationPowerSpectrum}
We already discussed that in order to completely describe photons (thus the CMB) we also need to account for polarization. It is therefore natural to define a power spectrum for the polarization, which can be done similarly as for the temperature.

We want to expand in spherical harmonics the Stokes parameters $Q$ and $U$\eqref{eq:stokes_parameters}, however we showed, in section \ref{sec:PhotonsPolarization} that under a rotation the combination $Q\pm iU$ will transform as a spin 2 fields. This means that we cannot resort to the usual spherical harmonics decomposition, instead we must use \textbf{spin-weighted spherical harmonics} $Y_{\ell m}^{\pm2}$. In this way we get 
$$Q(\versor n)\pm iU(\versor n) = \sum_{\ell m}a_{\ell m}^{\pm 2}Y_{\ell m}^{\pm2}(\versor n).$$
It is then common to use modes that are projectable on the regular spherical harmonics: we start defining
$$a^E_{\ell m}\defeq-\frac{a^2_{\ell m}+a^{-2}_{\ell m}}{2},\qquad a^B_{\ell m}\defeq\frac{a^2_{\ell m}-a^{-2}_{\ell m}}{2i},$$
then we can recompose these modes as
\begin{equation}\label{eq:EBmodes}
    E(\versor n)=\sum_{\ell m}a^E_{\ell m}Y_{\ell m}(\versor n),\qquad B(\versor n)=\sum_{\ell m}a^B_{\ell m}Y_{\ell m}(\versor n).
\end{equation}
The power spectra of the polarization can then be defined, as usual, as
\begin{align}\label{eq:PolPowerSpectrum}
    \langle a^E_{\ell m}a^{E*}_{\ell' m'}\rangle\defeq C_{\ell}^{EE}\delta_{\ell \ell'}\delta_{m m'},\\ \langle a^B_{\ell m}a^{B*}_{\ell' m'}\rangle\defeq C_{\ell}^{BB}\delta_{\ell \ell'}\delta_{m m'},\\\langle a_{\ell m}a^{E*}_{\ell' m'}\rangle\defeq C_{\ell}^{TE}\delta_{\ell \ell'}\delta_{m m'}.
\end{align}
\section{Tensor perturbations effects on the CMB}\label{sec:TensorPerturbations}
In the previous sections we focused on the effect of scalar perturbations on the anisotropies of the CMB. We will instead now study how tensor perturbations affects the evolution of these anisotropies.

In appendix \ref{app:tensorPerturbedLiouvilleOperator} we showed that, considering perturbed metric $$ds^2=-dt^2+a^2(\delta_{ij}h_{ij})dx^dx^j,$$the \emph{Liouville operator} reads as in \eqref{eq:tensorPerturbedLiouvilleOperator}
$$\hat{\mathbf{L}}[f]=\frac{\partial f}{\partial t}+\frac{\hat p^i}{a}\frac{\partial f}{\partial x^i}-\frac{1}{2}\frac{\partial f}{\partial t}\dot{h}_{ij}\hat p^i\hat p^j,$$
where $p^i=p\ \hat p^i$ is the local 3-momentum of a photon and $f$ the phase space distribution.

To obtain the equation describing the evolution of anisotropies we must expand the photon phase space distribution on a blackbody radiation background $\bar{f}(t,p)+\Upsilon(x^\mu,\mathbf p) $, assuming that the temperature is perturbed as $T(x^\mu,\versor p)=\bar T (1+\Theta(x^\mu,\versor{p}))$. In section \ref{sec:ThetaTimeEvolution} we showed that, in this way, the first order contribution to the phase space distribution reads $\Upsilon=-\Theta p\frac{\partial\bar f}{\partial p}$. This expansion allows to obtain a Liouville operator contribution at first order corresponding to 
$$\hat{\mathbf{L}}[\Upsilon]=-p\frac{\partial \bar f}{\partial p}\bigg[\frac{\partial \Theta}{\partial t}+\frac{\hat p^i}{a}\frac{\partial \Theta}{\partial x^i}+\frac{1}{2}\dot{h}_{ij}\hat p^i\hat p^j\bigg].$$
At this point we need to add the first order collision term associated to Compton scattering. For now let's consider the simplified form \eqref{eq:first_collision_term}. Using Boltzmann equation and canceling out the common factor $-p\frac{\partial\bar f}{\partial t}$ from both sides, as in section \ref{sec:ThetaTimeEvolution}, we get the differential equation that describes the time evolution of the CMB anisotropies in presence of tensor perturbations
\begin{equation}
    \label{eq:tensor_anisotropies_boltzmann}
    \frac{\partial \Theta}{\partial t}+\frac{\hat p^i}{a}\frac{\partial \Theta}{\partial x^i}+\frac{1}{2}\dot{h}_{ij}\hat p^i\hat p^j=n_{e}\sigma_T\bigg[\Theta_0-\Theta+\mathbf{\hat{p}}\cdot\mathbf{v}_{b}\bigg].
\end{equation}
\subsection{Coupling of tensors to anisotropies}
Previously, to solve equation \eqref{eq:phot_boltzmann_pert}, Fourier transformed $\Theta(t,\mathbf{x},\versor p)$ and then expanded in Legendre polynomial. However, this last step was justified, in section \ref{sec:BoltzmannMultipoleExpansion}, by noting that the equations depended only on the cosine of angle between the direction of motion of the photon $\versor p$ and the wave number vector of the Fourier transform $\mathbf{k}$. This corresponded to a rotational symmetry around one of the above vectors that implied that Legendre polynomials were the appropriate basis to expand on. We shall now study equation \eqref{eq:tensor_anisotropies_boltzmann}, and its symmetries, to understand  what will now be the right basis to use for this expansion.

To begin, let's recall that, being traceless transverse, in Fourier space, $h_{ij}$ can be separated in two independent polarizations
\begin{align*}
    0=\partial^ih_{ij}=\int\frac{d^3k}{(2\pi)^3}e^{\mathbf{k}\cdot\mathbf{x}}\tilde h_{ij}k^i \ \xrightarrow{\substack{\text{Traceless}\\\text{Symmetric}}}\ \tilde h_{ij}=\tilde h_{\boldsymbol \times} \mathbf{e}_{ij}^{\boldsymbol \times}+\tilde h_{\boldsymbol +} \mathbf{e}_{ij}^{\boldsymbol +}=
    \begin{pmatrix}
        \tilde h_{\boldsymbol \times} & \tilde h_{\boldsymbol +} & 0\\
        \tilde h_{\boldsymbol +} & -\tilde h_{\boldsymbol \times} & 0\\
        0&0&0
    \end{pmatrix}.
\end{align*}
The only term in \eqref{eq:tensor_anisotropies_boltzmann} that can induce more complicated angular dependence is $\dot{h}_{ij}\hat p^i\hat p^j$: considering spherical polar coordinates $(r,\theta,\phi)$ with $\mathbf k\parallel \versor z$, once Fourier transformed, this term will be proportional to
\begin{align*}
    \versor p=(\sin\theta\cos\phi,\sin\theta\sin\phi,\cos\theta)\quad \Rightarrow\quad (\mathbf{e}_{ij}^{\boldsymbol \times}+\mathbf{e}_{ij}^{\boldsymbol +})\hat p^i\hat p^j=\sin^2\theta(\cos2\phi+\sin2\phi).
\end{align*} 
This clearly shows that anisotropies coupled to tensor perturbations can no longer be decomposed on Legendre polynomials, since the azimuthal symmetry is now spoiled by the explicit dependence on $\phi$.

To individuate the appropriate basis for the spherical harmonics expansion, let's use the basis introduced by Hu and White in \cite{HuWhite}
\begin{equation}
    \label{eq:HuWhiteBasis}
    h_{ij}=-\sqrt{\frac{3}{2}}(h^{(+)}\mathbf{e}_{ij}^{(+)}+h^{(-)}\mathbf{e}_{ij}^{(-)}) \quad \text{with } \mathbf{e}^{(+)}=\begin{pmatrix}
        1&+i&0\\+i&-1&0\\0&0&0
    \end{pmatrix}\ \mathbf{e}^{(-)}=\begin{pmatrix}
        1&-i&0\\-i&-1&0\\0&0&0
    \end{pmatrix}.
\end{equation}
Comparing this definition and the previous basis, we can easily find the transformation between the two polarizations
\begin{align*}
    h_{\boldsymbol{+}}&=-\sqrt\frac{3}{2}(h^{(+)}+h^{(-)}),\qquad &h^{(+)}&=\frac{1}{\sqrt{6}}(h_{\boldsymbol{+}}-ih_{\boldsymbol{-}}),\\ h_{\boldsymbol{+}}&=-i\sqrt\frac{3}{2}(h^{(+)}-h^{(-)}),\qquad &h^{(-)}&=-\frac{1}{\sqrt{6}}(h_{\boldsymbol{+}}+ih_{\boldsymbol{-}}).
\end{align*}
Let's now project the versor $\versor p$, defined as above, onto $\mathbf{e}^{(\pm)}$, 
$$\mathbf{e}_{ij}^{(\pm)}\hat p^i\hat p^j=\sin^2\theta[\cos^2\phi-\sin^2\phi\pm2i\sin\phi\cos\phi]=\sin^2\theta e^{\pm i2\phi},$$
immediately we should recognize that this term is proportional to the spherical harmonics $Y_{2,\pm2}(\theta,\phi)=\frac{1}{4} \sqrt{\frac{15}{2\pi}} \, \sin^2\theta \, e^{ \pm i2\phi}$.\\This shows that the appropriate basis for the expansion of the anisotropies are the spherical harmonics $Y_{\ell m}$ with $m=\pm2$, since they all posses the same azimuthal symmetry $Y_{\ell m} \propto e^{\pm i2\phi}$ as the tensor term in \eqref{eq:tensor_anisotropies_boltzmann}.

Following Hu and White \cite{HuWhite} convention, we will use the following multipole expansion for the anisotropies
\begin{equation}
    \label{eq:TensorMultipoleExpansion}
    \Theta(t,\mathbf x,\versor p)=\int\frac{d^3k}{(2\pi)^3}e^{i\mathbf{k}\cdot\mathbf{x}}\sum_{\ell m} (-i)^{\ell}\sqrt{\frac{4\pi}{2\ell+1}}Y_{\ell m}(\versor p)\tilde\Theta_{\ell}^{(m)}(t,\mathbf k) ,
\end{equation}
in which we will only use the multipoles with $m=\pm2$, for the reasons we just explained.\\ Note that for $m=0$, since $Y_{\ell 0}=\sqrt{\frac{2\ell+1}{4\pi}}P_\ell(\cos\theta)$, we recover the Legendre polynomials expansion used for scalar perturbations. In this case the sum over the $2\ell+1$ different values of $m$ will result in the correspondence $\tilde \Theta_\ell^{(0)}=(2\ell+1)\tilde\Theta_{\ell}$.\\Lastly, we shall note that the normalization of the new polarization \eqref{eq:HuWhiteBasis} is such that 
\begin{align*}
    \frac{1}{2}\dot{\tilde{h}}_{ij}\hat p^i\hat p^j&=\frac{1}{2}\bigg(-\sqrt{\frac{3}{2}}\bigg)\bigg[\dot{\tilde{h}}^{(+)}\sin^2\theta e^{i2\phi}+\dot{\tilde{h}}^{(-)}\sin^2\theta e^{-i2\phi}\bigg]\\&=-\sqrt{\frac{4\pi}{5}}\bigg[\dot{\tilde{h}}^{(+)}Y_{2,2}(\versor p)+\dot{\tilde{h}}^{(-)}Y_{2,-2}(\versor p)\bigg],\\
\end{align*}
where the factor $-\sqrt{\frac{4\pi}{5}}$ precisely corresponds to $i^\ell\sqrt\frac{4\pi}{2\ell+1}|_{\ell=2}$, which is also the factor that get all the others $\tilde\Theta_2^{(2)}$ in the expansion. In this way the contribution of the tensor perturbation to the multipole expansion of equation \eqref{eq:tensor_anisotropies_boltzmann}, with $\ell=2$ and $m=\pm2$, will be exactly $\dot{\tilde{h}}^{(\pm)}$.
\subsection{Multipole expansion of tensor induced anisotropies}
Knowing that tensor perturbations are projected in the sky onto spherical harmonics with $m=\pm2$ while scalar perturbations are projected onto spherical harmonics with $m=0$, by their orthogonality, we can expand $\Theta$ in multipoles using only $Y_{\ell,\pm2}$. In this way we effectively decoupled the multipoles generated by tensor perturbations and the scalar ones.

First, we move to Fourier space where the equation describing anisotropies \eqref{eq:tensor_anisotropies_boltzmann} reads
$$\frac{\partial\tilde \Theta}{\partial t}+i\frac{k\cos\theta}{a} \tilde\Theta-\sqrt{\frac{4\pi}{5}}\bigg[\dot{\tilde{h}}^{(+)}Y_{2,2}(\versor p)+\dot{\tilde{h}}^{(-)}Y_{2,-2}(\versor p)\bigg]=n_{e}\sigma_T\bigg[\tilde\Theta_0-\tilde\Theta+k\tilde v_{b}\cos\theta\bigg],$$ in which we assumed that $\mathbf{v}_b$ is irrotational. \\Now, the expansion in spherical harmonics \eqref{eq:TensorMultipoleExpansion} yields 
$$\tilde{\Theta}_{\ell}^{(m)}(\mathbf{k})=i^\ell\sqrt{\frac{2\ell+1}{4\pi}}\int d\Omega_{\versor p}\tilde\Theta(\mathbf{x},\versor p)Y_{\ell,m}^*(\theta,\phi),$$
therefore, upon integration of the equation above we can decompose it in a set of ordinary differential equations, one for each multipole $\Theta_\ell^{(m)}$.\\
However, similarly to what happened in section \ref{sec:BoltzmannMultipoleExpansion}, we obtain from the second term on the left hand side a contribution proportional to $Y^*_{\ell m}(\theta,\phi)\cos\theta$. The proprieties of spherical harmonics allows to simplify this term as follows
$$
\cos\theta Y_{\ell m}(\theta,\phi)=\sqrt\frac{4\pi}{3}Y_{10}Y_{\ell m}=\sqrt{\frac{\ell^2-m^2}{2\ell-1}}Y_{\ell-1,m}+\sqrt{\frac{\ell^2-m^2}{2\ell+3}}Y_{\ell+1,m}
$$  and upon integration we therefore also get contributions from other multipoles.\\In this way equation \eqref{eq:tensor_anisotropies_boltzmann}
will be decomposed into
\begin{align}
    \dot{\tilde\Theta}_\ell^{(\pm2)}=\frac{k}{a}\bigg[\frac{\sqrt{\ell^2-m^2}}{2\ell-1}\Theta_{\ell-1}^{(\pm2)}-\frac{\sqrt{\ell^2-m^2}}{2\ell+3}\Theta_{\ell+1}^{(\pm2)}\bigg]-n_e\sigma_T\tilde\Theta_{\ell}^{(\pm2)}-\dot{\tilde h}^{(\pm)}\quad \ell\geq2,
\end{align}
where the contributions from $\tilde{\Theta_0}$ and $\tilde v_b$ are not appearing since they are multiplied by $m=0$ spherical harmonics.

As we discussed in section \ref{sec:ComptonPolarization} the Compton scattering is influenced by the polarization of incoming and outgoing photons. For this reason we must add corrections to the above equations.
To add the proper corrections we need also to describe the dynamics of the polarization.\\ Expanding the polarization in spherical harmonics as 
\begin{equation}
    (\Theta_Q\pm i\Theta_U)(\mathbf{x,\versor p})=\int\frac{d^3k}{(2\pi)^3}e^{i\mathbf{k}\cdot\mathbf{x}}\sum_{\ell m} (-i)^{\ell}\sqrt{\frac{4\pi}{2\ell+1}}Y^{\pm2}_{\ell m}(\versor p)(\tilde E_{\ell}^{(m)}\pm i\tilde B_\ell^{(m)}) ,\label{eq:EB_multipole_expansion_tensor}
\end{equation}
where $Y^{\pm2}_{\ell m}$ are again the spin weighted spherical harmonics. Again, arguing that the free streaming of photons is not influenced by the polarization while polarization is sourced by the quadrupole of the anisotropies, we can write
\begin{subequations}\label{eq:EB_multipole_equations_tensor}
    \begin{align}
        \dot{\tilde E}_\ell^{(\pm2)}&=\frac{k}{a}\bigg[\frac{\ell^2-2}{\ell^2(2\ell-1)}\tilde E_{\ell-1}^{(\pm2)}-\frac{2m}{\ell(\ell+1)}\tilde B_\ell^{(\pm2)}-\frac{\ell^2-2}{\ell^2(2\ell+3)}\tilde E_{\ell+1}^{(\pm2)}\bigg]+\nonumber\\&\qquad\qquad\qquad\qquad\qquad\qquad\qquad\qquad-n_e\sigma_T\bigg[\tilde E_{\ell}^{(\pm2)}+\sqrt{6}\Pi^{(\pm2)}\delta_{\ell,2}\bigg],\label{eq:E_multipole_equation_tensor}\\
        \dot{\tilde B}_\ell^{(\pm2)}&=\frac{k}{a}\bigg[\frac{\ell^2-2}{\ell^2(2\ell-1)}\tilde B_{\ell-1}^{(\pm2)}-\frac{2m}{\ell(\ell+1)}\tilde E_\ell^{(\pm2)}+\frac{\ell^2-2}{\ell^2(2\ell+3)}\tilde B_{\ell+1}^{(\pm2)}\bigg]+\nonumber\\&\qquad\qquad\qquad\qquad\qquad\qquad\qquad\qquad\qquad\qquad\qquad-n_e\sigma_T\tilde B_{\ell}^{(\pm2)},\label{eq:B_multipole_equation_tensor}\\
        \nonumber &\text{with}\quad\Pi^{(\pm2)}=\frac{1}{10}\bigg[\tilde\Theta_2^{(\pm2)}-\sqrt{6}\tilde E_2^{(\pm2)}\bigg].
    \end{align}
\end{subequations}
These are the equations governing the time evolution of the polarization in the plasma of photons considering only the multipoles coupled to the tensor perturbations.

In a similar way, also the equations for the anisotropies must be corrected, giving
\begin{align}\label{eq:Theta_multipole_equation_tensor}
    \dot{\tilde\Theta}_\ell^{(\pm2)}&=\frac{k}{a}\bigg[\frac{\sqrt{\ell^2-m^2}}{2\ell-1}\Theta_{\ell-1}^{(\pm2)}-\frac{\sqrt{\ell^2-m^2}}{2\ell+3}\Theta_{\ell+1}^{(\pm2)}\bigg]+\nonumber\\&\qquad\qquad\qquad\qquad\qquad+n_e\sigma_T\bigg[\Pi^{(\pm2)}\delta_{\ell,2}-\tilde\Theta_{\ell}^{(\pm2)}\bigg]-\dot{\tilde h}^{(\pm)}\quad \ell\geq2.
\end{align}
A full detailed derivation of these equations can be found in \cite{HuWhite}.

Let's stop for a second to appreciate that the equations \eqref{eq:Theta_multipole_equation_tensor}, \eqref{eq:E_multipole_equation_tensor} and \eqref{eq:B_multipole_equation_tensor} do not mix the $\pm2$ modes. Furthermore, for both values of $m$ the equations read the same, this means that from now on we can study only the $m=2$ modes and then use the results to obtain the $m=-2$ ones.
\section{Approximate solutions for the dynamics of the anisotropies}\label{sec:CMBApproximations}
All the differential equations we derived are pretty hare to solve analytically since they are strongly coupled and in theory they are an infinite number (as much as the number of multipoles). Usually we resort to numerical methods to obtain exact results, however some approximations can be useful to understand the general behavior of the CMB or even to simplify some numerical calculations.
\subsection{The tight coupling approximation}
\label{sec:TightCouplingApproximation}
At early times, when the plasma was denser and hotter, the mean free path of photons was very small and the rate of Compton scattering was very high. We will show that in this regime only the first two multipoles are  relevant to describe fully the plasma. This is somewhat similar to a fluid that can be fully described by its density and velocity field.

The guiding idea behind the tight coupling approximations is that the scatterings between baryons and photons, in this limit, is the only relevant interaction that determines the dynamics of the anisotropies. This is equivalent to consider the limit in which $n_e\sigma_T\gg1$, which means that the mean free path ($\propto\frac{1}{n_e\sigma_T}$) is very small. Starting from scalar perturbations, in equation \eqref{eq:multipole_boltzmann_photons_3} we can drop the time derivative, since it is negligible with respect to the terms multiplied by $n_e\sigma_T$. In this way we are left with
$$\frac{\ell k}{(2\ell+1)a}\tilde\Theta_{\ell-1}-n_e\sigma_T\bigg[\tilde\Theta_\ell-\frac{\delta_{\ell,2}}{10}\Pi\bigg]=-\frac{(\ell+1)k}{(2\ell+1)a}\tilde\Theta_{\ell+1},$$

from which we can note that the term $\Theta_{\ell+1}$ is small compared to $\Theta_{\ell-1}$. This essentially prove that only the first few moments are relevant while higher multipoles are always smaller and smaller as $\ell$ increases. In this limit we are left with two only differential equations, \eqref{eq:multipole_boltzmann_photons_0} and \eqref{eq:multipole_boltzmann_photons_2} (still coupled to the rest of the plasma), to be solved.\\ Similar considerations are valid for the polarization equations \eqref{eq:multipole_boltzmann_polatization_0} and \eqref{eq:multipole_boltzmann_polatization_2}, which can be simplified in the same way.

Also for tensor perturbations the tight coupling limit significantly simplifies the equations of motion. Reasoning as we have just done, from equations \eqref{eq:Theta_multipole_equation_tensor}, \eqref{eq:E_multipole_equation_tensor} and \eqref{eq:B_multipole_equation_tensor} we conclude that only the multipoles with $\ell=2$ are relevant. In this way, after ahving dropped the time derivatives, we are left with 
$$n_e\sigma_T\bigg[\frac{1}{10}\Pi^{(\pm2)}-\tilde\Theta_2^{(\pm2)}\bigg]\approx\dot{\tilde h}^{(\pm2)},\quad\tilde E_2^{(\pm2)}\approx-\sqrt{6}\Pi^{(\pm2)},\quad\tilde B_{2}^{(\pm2)}\approx0,$$
that using the definition of $\Pi^{(\pm2)}=\frac{1}{10}[\tilde\Theta_{2}^{(\pm2)}-\sqrt 6 \tilde E_{2}^{(\pm2)}]$ gives
\begin{equation}
    \label{eq:TightCouplingTensor}
    \tilde\Theta_2^{(\pm2)}\approx-\frac{4\dot{\tilde h}^{(\pm2)}}{3n_e\sigma_T},\qquad\tilde E_2^{(\pm2)}\approx-\frac{\sqrt{6}}{4}\tilde\Theta_2^{(\pm2)},\qquad\tilde B_{2}^{(\pm2)}\approx0.
\end{equation}
This approximation will be particularly useful in the next chapter to study the spectral distortions associated to the dissipation of gravitational waves.
\subsection{Improved tight coupling approximation}
To conclude we want to present a simple way to improve the tight coupling approximation relaxing the approximation of stationary solutions, without spoiling the reduced number of multipoles excited.\\ For the purpose of this work, we are going to focus primarily on tensor perturbations as illustrated in %\cite{chluba}.
Consider the equations \eqref{eq:Theta_multipole_equation_tensor}, \eqref{eq:E_multipole_equation_tensor} and \eqref{eq:B_multipole_equation_tensor} with conformal time  and considering only the quadrupole we have
\begin{align*}
    \partial_{\tau}\tilde\Theta_{2}^{(\pm2)}&=n_e\sigma_T a\bigg[\frac{9}{10} \tilde\Theta_{2}^{(\pm2)}+\frac{\sqrt{6}}{10}\tilde E_{2}^{(\pm2)}\bigg]-\partial_{\tau}\tilde h^{(\pm)},\\
    \partial_{\tau}\tilde E_{2}^{(\pm2)}&=n_e\sigma_T a\bigg[\frac{2}{5} \tilde E_{2}^{(\pm2)}+\frac{\sqrt{6}}{10}\tilde \Theta_{2}^{(\pm2)}\bigg]-k\frac{2}{3}\tilde B_2^{\pm2},\\
    \partial_{\tau}\tilde B_{2}^{(\pm2)}&=n_e\sigma_T a\tilde B_{2}^{(\pm2)}+k\frac{2}{3}\tilde E_2^{\pm2}.
\end{align*}
To solve these equations we should proceed with an ansatz: assume that the solution has the form $ \tilde\Theta_{2}^{(2)}=A_\Theta e^{ik\tau}$, $\tilde E_{2}^{(2)}=A_E e^{ik\tau}$ and $\tilde B_{2}^{(2)}=A_B e^{ik\tau}$ and that the gravitational perturbation is $\tilde h^{(\pm)}=A_he^{ik\tau}$, where we dropped the $\pm$ since the equations are the same for both cases.\\ In this way the above system of differential equations reduces to a system of linear equations for the coefficients
\begin{align*}
    ikA_\Theta&=n_e\sigma_T a\bigg[\frac{9}{10} A_\Theta+\frac{\sqrt{6}}{10}A_E\bigg]-ik A_h,\\
    ikA_E&=n_e\sigma_T a\bigg[\frac{2}{5} A_E+\frac{\sqrt{6}}{10}A_\Theta\bigg]-k\frac{2}{3}A_B,\\
    ikA_B&=n_e\sigma_T a A_B+k\frac{2}{3}A_E.
\end{align*}
Once solved this system we find
\begin{subequations}\label{eq:improved_TightCouplingApproximation}
    \begin{align}
        \frac{|A_\Theta|}{\frac{4}{3}\frac{|A_h|}{n_e\sigma_T a}}&=\sqrt\frac{1+\frac{341}{36}\xi^2+\frac{625}{324}\xi^4}{1+\frac{142}{9}\xi^2+\frac{1649}{82}\xi^4+\frac{2500}{729}\xi^6},\nonumber\\\tan\phi_\Theta&=-\frac{11}{6}\xi\frac{1+\frac{697}{99}\xi^2+\frac{1250}{891}\xi^4}{1+\frac{197}{18}\xi^2+\frac{125}{54}\xi^4},\\
        \frac{|A_E|}{\frac{4}{3}\frac{|A_h|}{n_e\sigma_T a}}&=\frac{\sqrt 6}{4}\sqrt\frac{1+\xi^2}{1+\frac{142}{9}\xi^2+\frac{1649}{82}\xi^4+\frac{2500}{729}\xi^6},\nonumber\\\tan(\phi_E-\pi)&=-\frac{13}{3}\xi\frac{1+\frac{121}{117}\xi^2}{1-\xi^2-\frac{50}{27}\xi^4},\\
        \frac{|A_B|}{\frac{4}{3}\frac{|A_h|}{n_e\sigma_T a}}&=\frac{\xi}{\sqrt 6}\sqrt\frac{1}{1+\frac{142}{9}\xi^2+\frac{1649}{82}\xi^4+\frac{2500}{729}\xi^6}\nonumber\\\tan(\phi_B-\pi)&=-\frac{16}{3}\xi\frac{1+\frac{121}{117}\xi^2}{1-\frac{19}{3}\xi^2}.
    \end{align}
\end{subequations}
In the above $\xi\defeq\frac{k}{n_e\sigma_T a}$, note that in the tight coupling limit $\xi\ll1$ and the equations above reduce to the previous discussed approximations with $\phi_E=\phi_B=\phi_\Theta=0$.\\ This approximation also shows that as we exit the thigh coupling regimes and photons start to free stream ($\xi\gg1$) the anisotropies start to decay  as $\xi^{-1}$, while the polarization decays even faster as $\xi^{-2}$.\\